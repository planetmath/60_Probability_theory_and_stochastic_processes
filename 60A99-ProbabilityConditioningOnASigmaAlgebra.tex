\documentclass[12pt]{article}
\usepackage{pmmeta}
\pmcanonicalname{ProbabilityConditioningOnASigmaAlgebra}
\pmcreated{2013-03-22 16:25:05}
\pmmodified{2013-03-22 16:25:05}
\pmowner{CWoo}{3771}
\pmmodifier{CWoo}{3771}
\pmtitle{probability conditioning on a sigma algebra}
\pmrecord{7}{38568}
\pmprivacy{1}
\pmauthor{CWoo}{3771}
\pmtype{Definition}
\pmcomment{trigger rebuild}
\pmclassification{msc}{60A99}
\pmclassification{msc}{60A10}

\endmetadata

\usepackage{amssymb,amscd}
\usepackage{amsmath}
\usepackage{amsfonts}

% used for TeXing text within eps files
%\usepackage{psfrag}
% need this for including graphics (\includegraphics)
%\usepackage{graphicx}
% for neatly defining theorems and propositions
%\usepackage{amsthm}
% making logically defined graphics
%%\usepackage{xypic}
\usepackage{pst-plot}
\usepackage{psfrag}

% define commands here

\begin{document}
Let $(\Omega, \mathfrak{B}, \mu)$ be a probability space and $B\in \mathfrak{B}$ an event.  Let $\mathfrak{D}$ be a sub sigma algebra of $\mathfrak{B}$.  The \emph{\PMlinkescapetext{conditional probability} of $B$ given $\mathfrak{D}$} is defined to be the conditional expectation of the random variable $1_B$ defined on $\Omega$, given $\mathfrak{D}$.  We denote this conditional probability by $\mu(B|\mathfrak{D}):=E(1_B| \mathfrak{D})$.  $1_B$ is also known as the indicator function.  

Similarly, we can define a conditional probability given a random variable.  Let $X$ be a random variable defined on $\Omega$.  The \emph{conditional probability of $B$ given $X$} is defined to be $\mu(B|\mathfrak{B}_X)$, where $\mathfrak{B}_X$ is the sub sigma algebra of $\mathfrak{B}$, \PMlinkname{generated by}{MathcalFMeasurableFunction} $X$.  The conditional probability of $B$ given $X$ is simply written $\mu(B|X)$.

\textbf{Remark}.  Both $\mu(B|\mathfrak{D})$ and $\mu(B|X)$ are random variables, the former is $\mathfrak{D}$-measurable, and the latter is $\mathfrak{B}_X$-measurable.
%%%%%
%%%%%
\end{document}
