\documentclass[12pt]{article}
\usepackage{pmmeta}
\pmcanonicalname{ProofOfDoobsInequalities}
\pmcreated{2013-03-22 18:39:55}
\pmmodified{2013-03-22 18:39:55}
\pmowner{gel}{22282}
\pmmodifier{gel}{22282}
\pmtitle{proof of Doob's inequalities}
\pmrecord{5}{41409}
\pmprivacy{1}
\pmauthor{gel}{22282}
\pmtype{Proof}
\pmcomment{trigger rebuild}
\pmclassification{msc}{60G46}
\pmclassification{msc}{60G44}
\pmclassification{msc}{60G42}
%\pmkeywords{submartingale}

% almost certainly you want these
\usepackage{amssymb}
\usepackage{amsmath}
\usepackage{amsfonts}

% used for TeXing text within eps files
%\usepackage{psfrag}
% need this for including graphics (\includegraphics)
%\usepackage{graphicx}
% for neatly defining theorems and propositions
\usepackage{amsthm}
% making logically defined graphics
%%%\usepackage{xypic}

% there are many more packages, add them here as you need them

% define commands here
\newtheorem*{theorem*}{Theorem}
\newtheorem*{lemma*}{Lemma}
\newtheorem*{corollary*}{Corollary}
\newtheorem*{definition*}{Definition}
\newtheorem{theorem}{Theorem}
\newtheorem{lemma}{Lemma}
\newtheorem{corollary}{Corollary}
\newtheorem{definition}{Definition}

\begin{document}
\PMlinkescapeword{index set}
\PMlinkescapeword{term}

Let $(\Omega,\mathcal{F},(\mathcal{F}_t)_{t\in\mathbb{T}},\mathbb{P})$ be a filtered probability space with countable index set $\mathbb{T}$.
If $(X_t)_{t\in\mathbb{T}}$ is a submartingale, we show that
\begin{equation}\label{eq:1}
\mathbb{P}\left(\sup_{s\le t}X_s\ge K\right)\le K^{-1}\mathbb{E}[(X_t)_+]
\end{equation}
and if $X$ is a martingale or nonnegative submartingale then,
\begin{gather}
\label{eq:2}\mathbb{P}(X^*_t\ge K)\le K^{-1}\mathbb{E}[|X_t|],\\
\label{eq:3}\Vert X^*_t\Vert_p\le \frac{p}{p-1}\Vert X_t\Vert_p. 
\end{gather}
for every $K>0$ and $p>1$.

First, let us consider the case where $\mathbb{T}$ is finite. The first time at which $X_t\ge K$,
\begin{equation*}
\tau=\inf\left\{t\in\mathbb{T}:X_t\ge K\right\}
\end{equation*}
is a stopping time (as hitting times are stopping times). By Doob's optional sampling theorem for submartingales $X_{\tau\wedge t}\le\mathbb{E}[X_t\mid\mathcal{F}_{\tau\wedge t}]$ and therefore,
\begin{equation*}
K\mathbb{P}(\tau\le t)
\le\mathbb{E}[1_{\{\tau\le t\}}X_{\tau\wedge t}]
\le\mathbb{E}[1_{\{\tau\le t\}}X_{t}]
\end{equation*}
However, $\tau\le t$ if and only if $\sup_{s\le t}X_s\ge K$ giving,
\begin{equation}\label{eq:4}
\mathbb{P}\left(\sup_{s\le t}X_s\ge K\right)\le K^{-1}\mathbb{E}[1_{\{\sup_{s\le t}X_s\ge K\}}X_t],
\end{equation}
where the supremum is understood to be over $s\in\mathbb{T}$.
Now suppose that $\mathbb{T}$ is countable. Then choose finite subsets $\mathbb{T}_n\subseteq\mathbb{T}$ which increase to $\mathbb{T}$ as $n$ goes to infinity. Replacing $\mathbb{T}$ by $\mathbb{T}_n$ in inequality (\ref{eq:4}) and using the monotone convergence theorem to take the limit $n\rightarrow\infty$ extends (\ref{eq:4}) to arbitrary uncountable index sets. Then, inequality (\ref{eq:1}) follows immediately from (\ref{eq:4}).

Now, suppose that $X$ is a martingale. Jensen's inequality gives
\begin{equation*}
\mathbb{E}[|X_t|\mid\mathcal{F}_s]\ge \left|\mathbb{E}[X_t\mid\mathcal{F}_s]\right|=|X_s|
\end{equation*}
for any $s<t$, so $|X|$ is a nonnegative submartingale. Therefore, it is enough to prove inequalities (\ref{eq:2}) and (\ref{eq:3}) for $X$ a nonnegative submartingale, and the martingale case follows by replacing $X$ by $|X|$.

So, we take $X$ to be a nonnegative submartingale in the following. In this case, (\ref{eq:2}) just reduces to (\ref{eq:1}) and it only remains to prove inequality (\ref{eq:3}).

For $p>1$, multiply (\ref{eq:4}) by $K^{p-1}$ and integrate up to some limit $L>0$,
\begin{equation}\label{eq:5}
\int_0^LK^{p-1}\mathbb{P}(X^*_t\ge K)\,dK\le \int_0^LK^{p-2}\mathbb{E}[1_{\{X^*_t\ge K\}}X_t]\,dK.
\end{equation}
The left hand side of this inequality can be computed by commuting the order of integration with respect to $\mathbb{P}$ and $dK$ (Fubini's theorem),
\begin{equation*}\begin{split}
\int_0^L K^{p-1}\mathbb{P}(X^*_t\ge K)\,dK
&=\mathbb{E}\left[\int_0^L K^{p-1}1_{\{X^*_t\ge K\}}\,dK\right]\\
&=\frac{1}{p}\mathbb{E}[(L\wedge X^*)^p].
\end{split}\end{equation*}
The right hand side of (\ref{eq:5}) can be computed similarly,
\begin{equation*}\begin{split}
\int_0^LK^{p-2}\mathbb{E}[1_{\{X^*_t\ge K\}}X_t]\,dK
&=\mathbb{E}\left[X_t\int_0^LK^{p-2}1_{\{X^*_t\ge K\}}\,dK\right]\\
&=\frac{1}{p-1}\mathbb{E}[X_t(L\wedge X^*_t)^{p-1}].
\end{split}\end{equation*}
Putting these back into (\ref{eq:5}),
\begin{equation}\label{eq:6}
\Vert L\wedge X^*_t\Vert_p^p\le\frac{p}{p-1}\mathbb{E}[X_t(L\wedge X^*_t)^{p-1}].
\end{equation}
Now let $q=p/(p-1)$, so that $p,q$ are \PMlinkname{conjugate}{ConjugateIndex} and the H\"older inequality gives
\begin{equation*}
\mathbb{E}[X_t(L\wedge X^*_t)^{p-1}]
\le\Vert X_t\Vert_p \Vert (L\wedge X^*_t)^{p-1}\Vert_q
=\Vert X_t\Vert_p \Vert L\wedge X^*_t\Vert_p^{p-1}.
\end{equation*}
Substituting into (\ref{eq:6}), the finite term $\Vert L\wedge X^*_t\Vert_p^{p-1}$ cancels to get
\begin{equation*}
\Vert L\wedge X^*_t \Vert_p\le \frac{p}{p-1}\Vert X_t\Vert_p,
\end{equation*}
and the result follows by letting $L$ increase to infinity.

%%%%%
%%%%%
\end{document}
