\documentclass[12pt]{article}
\usepackage{pmmeta}
\pmcanonicalname{ConditionalExpectationUnderChangeOfMeasure}
\pmcreated{2013-03-22 16:54:21}
\pmmodified{2013-03-22 16:54:21}
\pmowner{stevecheng}{10074}
\pmmodifier{stevecheng}{10074}
\pmtitle{conditional expectation under change of measure}
\pmrecord{9}{39165}
\pmprivacy{1}
\pmauthor{stevecheng}{10074}
\pmtype{Derivation}
\pmcomment{trigger rebuild}
\pmclassification{msc}{60A10}
\pmclassification{msc}{60-00}
\pmrelated{Martingale}
\pmrelated{ConditionalExpectation}

\endmetadata

% The standard font packages
\usepackage{amssymb}
\usepackage{amsmath}
\usepackage{amsfonts}

% For neatly defining theorems and definitions
\usepackage{amsthm}

% Including EPS/PDF graphics (\includegraphics)
%\usepackage{graphicx}

% Making matrix-based graphics
%%%\usepackage{xypic}

% Enumeration lists with different styles
%\usepackage{enumerate}

% Set up the theorem environments
\newtheorem{thm}{Theorem}
%\newtheorem*{thm*}{Theorem}

\providecommand{\defnterm}[1]{\emph{#1}}

% Probability stuff
\newcommand{\PP}{\mathbb{P}}
\newcommand{\E}{\mathbb{E}}
\newcommand{\EQ}{\mathbb{E}^\mathbb{Q}}
\newcommand{\PQ}{\mathbb{Q}}
\newcommand{\sF}{\mathcal{F}}
\newcommand{\sG}{\mathcal{G}}

\begin{document}

Let $\PP$ be a given probability measure
on some $\sigma$-algebra $\sF$.
Suppose a new probability measure $\PQ$ is defined
by $d\PQ = Z \, d\PP$, using some $\sF$-measurable 
random variable $Z$
as the Radon-Nikodym derivative. 
(Necessarily we must have $Z \geq 0$ almost surely, and $\E Z = 1$.)

We denote with $\E$ the expectation with respect to the measure $\PP$,
and with $\EQ$ the expectation with respect to the measure $\PQ$.

\begin{thm}
If $\PQ$ is restricted to a sub-$\sigma$-algebra
$\sG \subseteq \sF$,
then the restriction has the conditional expectation
$\E[Z \mid \sG]$ as its Radon-Nikodym derivative:
$d \PQ_{\mid\sG} = \E[ Z \mid \sG ] \, d\PP_{\mid \sG}$.

In other words,
\[
\frac{d\PQ_{\mid \sG}}{d\PP_{\mid \sG}} = \left( \frac{d\PQ}{d\PP}  \right)_{\mid \sG}\,.
\]


\begin{proof}
It is required to prove that, for all $B \in \sG$,
\[
\PQ(B) = \E \bigl[ \E [ Z \mid \sG] \, 1_B  \bigr]\,.
\]
But this follows at once from the law of iterated conditional expectations:
\[
\E \bigl [ \E [ Z \mid \sG ] \, 1_B \bigr]
= 
\E \bigl [ \E [ Z 1_B \mid \sG ] \bigr]
= \E [Z 1_B] = \PQ(B) \,. \qedhere
\]
\end{proof}
\end{thm}


\begin{thm}
Let $\sG \subseteq \sF$ be any sub-$\sigma$-algebra.
For any $\sF$-measurable random variable $X$,
\[
\E [ Z \mid \sG ] \, \EQ [ X \mid \sG]
= \E[ ZX \mid \sG]\,.
\]
That is,
\[
\left(\frac{d\PQ}{d\PP}\right)_{\mid \sG} \,
\EQ [ X \mid \sG ] = 
\E \left[ \frac{d\PQ}{d\PP} \, X \mid \sG \right]\,.
\]

\begin{proof}
Let $Y = \E[Z \mid \sG]$, and 
$B \in \sG$.  We find:
\begin{align*}
\EQ \bigl[ 1_B \, \E[ZX \mid \sG] \bigr] &=
\E \bigl[ Y 1_B \, \E [ ZX \mid \sG] \bigr]
& \text{(since $d \PQ_{\mid \sG} = Y \, d \PP_{\mid \sG}$)} \\
&= \E \bigl[ \E[ Y 1_B \, ZX \mid \sG] \bigr] \\
&= \E [Y 1_B \, ZX ] \\
&= \EQ [ Y 1_B \, X ] & \text{(since $d\PQ = Z \, d\PP$)} \\
&= \EQ \bigl[  1_B \, \EQ[ Y X \mid \sG] \bigr]\,.
\end{align*}
Since $B \in \sG$ is arbitrary, we can equate the $\sG$-measurable integrands:
\[
\E [ ZX \mid \sG ] = \EQ [ YX \mid \sG] = Y \EQ [X \mid \sG]\,.
\qedhere
\]
\end{proof}
\end{thm}

Observe that if $d \PQ / d \PP > 0$ almost surely,
then
\[
\EQ[X \mid \sG] = \E\left [ \frac{d\PQ}{d\PP} X \mid \sG \right]
\Big/ \left( \frac{d\PQ}{d\PP} \right)_{\mid \sG}\,.
\]

\begin{thm}
If $X_t$ is a martingale with respect to $\PQ$ and some
filtration $\{ \sF_t \}$,
then $X_t Z_t$ is a martingale with respect to $\PP$ and $\{ \sF_t\}$,
where $Z_t = \E [ Z \mid \sF_t]$.

\begin{proof}
First observe that $X_t Z_t$ is indeed $\sF_t$-measurable.
Then, we can apply Theorem 2,
with $X$ in the statement of that theorem replaced by
$X_t$, 
$Z$ replaced by $Z_t$, 
$\sF$ replaced by $\sF_t$, and $\sG$ replaced by $\sF_s$ ($s \leq t$),
to obtain:
\[
\E[ X_t Z_t \mid \sF_s ] = Z_s \, \EQ[ X_t \mid \sF_s]
= Z_s X_s\,,
\]
thus proving that $X_t Z_t$ is a martingale under $\PP$
and $\{ \sF_t \}$.
\end{proof}
\end{thm}

Sometimes the random variables $Z_t$ in Theorem 3
are written as $\left( \frac{d\PQ}{d\PP} \right)_t$.
(This is a Radon-Nikodym derivative \emph{process};
note that $Z_t$ defined as $Z_t = \E[Z \mid \sF_t]$
 is always a martingale
under $\PP$ and $\{ \sF_t \}$.)

Under the hypothesis $Z_t > 0$, 
there is an alternate restatement of Theorem 3
that may be more easily remembered:

\begin{thm}
Let $Z_t = ( d\PQ / d\PP )_t > 0$ almost surely.
Then $X_t$ is a martingale with respect to $\PP$,
if and only if $X_t / Z_t$ is a martingale with respect to $\PQ$.
\end{thm}

%%%%%
%%%%%
\end{document}
