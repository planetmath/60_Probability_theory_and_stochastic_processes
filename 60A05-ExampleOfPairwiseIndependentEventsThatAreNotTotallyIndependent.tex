\documentclass[12pt]{article}
\usepackage{pmmeta}
\pmcanonicalname{ExampleOfPairwiseIndependentEventsThatAreNotTotallyIndependent}
\pmcreated{2013-03-22 13:38:51}
\pmmodified{2013-03-22 13:38:51}
\pmowner{bbukh}{348}
\pmmodifier{bbukh}{348}
\pmtitle{example of pairwise independent events that are not totally independent}
\pmrecord{6}{34299}
\pmprivacy{1}
\pmauthor{bbukh}{348}
\pmtype{Example}
\pmcomment{trigger rebuild}
\pmclassification{msc}{60A05}

\usepackage{amssymb}
\usepackage{amsmath}
\usepackage{amsfonts}
\begin{document}
\PMlinkescapeword{blue}
\PMlinkescapeword{red}
\PMlinkescapeword{white}
\PMlinkescapeword{components}
Consider a fair tetrahedral die whose sides are painted red, green, blue, and white. Roll the die. Let $X_r, X_g, X_b$ be the events that die falls on a side that have red, green, and blue color components, respectively. Then
\begin{align*}
P(X_r)=P(X_g)&=P(X_b)=\frac{1}{2},\\
P(X_r \cap X_g)=P(X_w)&=\frac{1}{4}=P(X_r)P(X_g),\\
\intertext{but}
P(X_r \cap X_g \cap X_b)=\frac{1}{4}&\neq \frac{1}{8}=P(X_r)P(X_g)P(X_b).
\end{align*}
%%%%%
%%%%%
\end{document}
