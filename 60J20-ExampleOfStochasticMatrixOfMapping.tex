\documentclass[12pt]{article}
\usepackage{pmmeta}
\pmcanonicalname{ExampleOfStochasticMatrixOfMapping}
\pmcreated{2014-04-28 3:33:09}
\pmmodified{2014-04-28 3:33:09}
\pmowner{rspuzio}{6075}
\pmmodifier{PMBookProject}{1000683}
\pmtitle{Example of stochastic matrix of mapping}
\pmrecord{21}{88100}
\pmprivacy{1}
\pmauthor{rspuzio}{1000683}
\pmtype{Example}

% this is the default PlanetMath preamble.  as your knowledge
% of TeX increases, you will probably want to edit this, but
% it should be fine as is for beginners.

% almost certainly you want these
\usepackage{amssymb}
\usepackage{amsmath}
\usepackage{amsfonts}

% need this for including graphics (\includegraphics)
\usepackage{graphicx}
% for neatly defining theorems and propositions
\usepackage{amsthm}

% making logically defined graphics
%\usepackage{xypic}
% used for TeXing text within eps files
%\usepackage{psfrag}

% there are many more packages, add them here as you need them

% define commands here

\newcommand{\vecify}{{\mathcal V}}
\newcommand{\Act}{{A}}
\newcommand{\act}{{a}}
\newcommand{\Sit}{{S}}
\newcommand{\occ}{{v}}
\newcommand{\univ}{{\mathbf D}}
\newcommand{\uout}{{d_{out}}}
\newcommand{\uin}{{d_{in}}}
\newcommand{\mangle}{{\mathbf C}}

\newcommand{\psheaf}{{\mathcal F}}
\newcommand{\scat}{{\mathtt{Stoch}}}
\newcommand{\subs}{{\mathtt{Sys}}}
\newcommand{\mcat}{{\mathtt{Meas}}}
\newcommand{\eop}{{$\blacksquare$}}
\newcommand{\eod}{{${}$\\}}
\newcommand{\bra}{{\langle}}
\newcommand{\ket}{{\rangle}}

\newcommand{\cN}{{\mathcal N}}
\newcommand{\bR}{{\mathbb R}}
\newcommand{\fm}{{\mathfrak m}}
\newcommand{\cP}{{\mathcal P}}

\begin{document}
In order to understand the notion of stochastic matrix associated
to a mapping and its dual, we will work through a simple example.
Let $X = \{a,b,c\}$ and let $Y = \{d,e\}$, and define the mapping
$f \colon X \to Y$ as follows:

\begin{align*}
 f(a) &= d \\
 f(b) &= d \\
 f(c) &= e
\end{align*}

Then $\vecify X$ is a 3-dimensional real vector space with basis
\[
 \delta_a = \begin{pmatrix} 1 \cr 0 \cr 0 \end{pmatrix}, \qquad
 \delta_b = \begin{pmatrix} 0 \cr 1 \cr 0 \end{pmatrix}, \qquad
 \delta_c = \begin{pmatrix} 0 \cr 0 \cr 1 \end{pmatrix}
\]
and $\vecify Y$ is a 3-dimensional real vector space with basis
\[
 \delta_c = \begin{pmatrix} 1 \cr 0 \end{pmatrix}, \qquad
 \delta_d = \begin{pmatrix} 0 \cr 1 \end{pmatrix}
\]
and
\[
 {\vecify f} = \begin{pmatrix} 1 & 1 & 0 \cr 0 & 0 & 1 \end{pmatrix} .
\]

To form the dual, we first renormalize the rows to sum to unity, 
then transpose:
\[
 \begin{pmatrix} 1 & 1 & 0 \cr 
                 0 & 0 & 1 \end{pmatrix} \xrightarrow{ren}
 \begin{pmatrix} \frac{1}{2} & \frac{1}{2} & 0 \cr 
                 0 & 0 & 1 \end{pmatrix} \xrightarrow{*}
 \begin{pmatrix} \frac{1}{2} & 0 \cr
                 \frac{1}{2} & 0 \cr
                 0 & 1 \end{pmatrix}
\]

Next, to illustrate inclusions, we shall examine the map $i \colon Y 
\hookrightarrow X$ defined as follows:

\begin{align*}
 f(d) &=& a \\
 f(e) &=& b
\end{align*}

Following the same procedures as above, for this map we find that
\[
 {\vecify i} = \begin{pmatrix} 1 & 0 \\ 0 & 1 \\ 0 & 0 \end{pmatrix}
\]
and
\[
  ({\vecify i})^\natural = 
   \begin{pmatrix} 1 & 0 & 0 \\ 0 & 1 & 0 \end{pmatrix}
\]

\end{document}
