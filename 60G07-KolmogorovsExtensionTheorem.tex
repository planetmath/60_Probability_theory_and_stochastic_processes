\documentclass[12pt]{article}
\usepackage{pmmeta}
\pmcanonicalname{KolmogorovsExtensionTheorem}
\pmcreated{2013-04-12 21:33:32}
\pmmodified{2013-04-12 21:33:32}
\pmowner{Filipe}{28191}
\pmmodifier{Filipe}{28191}
\pmtitle{Kolmogorov's extension theorem}
\pmrecord{3}{87348}
\pmprivacy{1}
\pmauthor{Filipe}{28191}
\pmtype{Theorem}
\pmclassification{msc}{60G07}

% this is the default PlanetMath preamble.  as your knowledge
% of TeX increases, you will probably want to edit this, but
% it should be fine as is for beginners.

% almost certainly you want these
\usepackage{amssymb}
\usepackage{amsmath}
\usepackage{amsfonts}

% need this for including graphics (\includegraphics)
\usepackage{graphicx}
% for neatly defining theorems and propositions
\usepackage{amsthm}

% making logically defined graphics
%\usepackage{xypic}
% used for TeXing text within eps files
%\usepackage{psfrag}

% there are many more packages, add them here as you need them

% define commands here

\begin{document}
For all $t_1,\cdots,t_k$, $k\in \mathbb{N}$, let $v_{t_1,\cdots,t_k}$ be probability measures on $\mathbb{R}^{nk}$ satisfying the following properties (consistency conditions):
\begin{enumerate}
\item $v_{t_{\sigma(1)},\cdots,t_{\sigma(k)}}(F_1\times \cdots \times F_k)=v_{t_1,\cdots t_k}(F_{\sigma^{-1}(1)}\times \cdots F_{\sigma^{-1}(k)})$ for all permutations $\sigma$ of $\lbrace 1,2,\cdots,k\rbrace$ and for all Borel sets $F_i$ of $\mathbb{R}^n$
\item $v_{t_1,\cdots,t_k}(F_1 \times \cdots \times F_k)=v_{t_1,\cdots,t_k,t_{k+1},\cdots t_{k+m}}(F_1 \times \cdots \times F_k \times \mathbb{R}^n \times \cdots \times \mathbb{R}^n)$ for all $m \in \mathbb{N}$ and for all Borel sets $F_i$ of $\mathbb{R}^n$
\end{enumerate}

Then there exists a probability space $(\Omega,\mathcal{F},P)$ and a stochastic process $X_t$ on $\Omega$, indexed by $T$, taking values in $\mathbb{R}^n$ such that 
$$ v_{t_1,\cdots,t_k}(F_1 \times \cdots \times F_k)=P(X_{t_1}\in F_1, \cdots, X_{t_k}\in F_k)$$ for all $t_i\in T, k \in \mathbb{R}^n$ and all Borel sets $F_i$ of $\mathbb{R}^n$
\end{document}
