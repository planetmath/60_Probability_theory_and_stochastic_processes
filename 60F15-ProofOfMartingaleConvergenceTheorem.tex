\documentclass[12pt]{article}
\usepackage{pmmeta}
\pmcanonicalname{ProofOfMartingaleConvergenceTheorem}
\pmcreated{2013-03-22 18:34:33}
\pmmodified{2013-03-22 18:34:33}
\pmowner{scineram}{4030}
\pmmodifier{scineram}{4030}
\pmtitle{proof of martingale convergence theorem}
\pmrecord{4}{41300}
\pmprivacy{1}
\pmauthor{scineram}{4030}
\pmtype{Proof}
\pmcomment{trigger rebuild}
\pmclassification{msc}{60F15}
\pmclassification{msc}{60G44}
\pmclassification{msc}{60G46}
\pmclassification{msc}{60G42}

% this is the default PlanetMath preamble.  as your knowledge
% of TeX increases, you will probably want to edit this, but
% it should be fine as is for beginners.

% almost certainly you want these
\usepackage{amssymb}
\usepackage{amsmath}
\usepackage{amsfonts}

% used for TeXing text within eps files
%\usepackage{psfrag}
% need this for including graphics (\includegraphics)
%\usepackage{graphicx}
% for neatly defining theorems and propositions
\usepackage{amsthm}
% making logically defined graphics
%%%\usepackage{xypic}

% there are many more packages, add them here as you need them

% define commands here

\begin{document}
Let $(X_n)_{n\in\mathbb{N}}$ be a supermartingale such that $\mathbb{E}|X_n|\le M$, and let $a<b$. We define a random variable counting how many times the process crosses the stripe between $a$ and $b$:
\[U_n:=\max\{0,r\in\{1,\ldots,n\}\mid\exists 0\le s_1<t_1<s_2<t_2<\ldots<s_r<t_r\le n\;\forall i\in\{1,\ldots,r\}\colon X_{s_i}\le a\wedge X_{t_i}\ge b\}.\]
Obviously $U_{n+1}\ge U_n$ therefore $U_\infty=\lim_{n\to\infty}U_n$ exists almost surely. Next we will construct a new process that mirrors the movement of $X_n$ but only if the original process is underway of going from below $a$ to over $b$, and is constant otherwise. To do this let $C_1:=\chi[X_0<a]$, $C_k:=\chi[C_{k-1}=0\wedge X_{k-1}<a]+\chi[C_{k-1}=1\wedge X_{k-1}\le b]$ for $k\ge 2$, and define $Y_0:=0$, $Y_n:=\sum\limits_{k=1}^n C_k(X_k-X_{k-1})$. Then $Y_n$ is also a supermartingale, and the inequality $Y_n\ge(b-a)U_n-|X_n-a|$ holds, which gives $0\ge\mathbb{E}(Y_n)\ge(b-a)\mathbb{E}(U_n)-\mathbb{E}|X_n-a|$. After rearrangement we get
\[\mathbb{E}(U_n)\le\frac{\mathbb{E}|X_n-a|}{b-a}\le\frac{\mathbb{E}|X_n|+|a|}{b-a}\le\frac{M+|a|}{b-a}.\]
Therefore by the monotone convergence theorem
\[\mathbb{E}(U_\infty)=\lim_{n\to\infty}\mathbb{E}(U_n)\le\frac{M+|a|}{b-a}<\infty,\]
which means $\mathbb{P}(U_\infty=\infty)=0$. Since $a$ and $b$ were arbitrary $X=\lim_{n\to\infty}X_n$ exists almost surely. Now the Fatou lemma gives
\[\mathbb{E}|X|=\mathbb{E}(\lim_{n\to\infty}|X_n|)\le\liminf_{n\to\infty}\mathbb{E}|X_n|\le M<\infty.\]
Thus $X_n$ is in fact convergent almost surely, and $X\in L^1. \qed$
%%%%%
%%%%%
\end{document}
