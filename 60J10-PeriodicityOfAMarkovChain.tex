\documentclass[12pt]{article}
\usepackage{pmmeta}
\pmcanonicalname{PeriodicityOfAMarkovChain}
\pmcreated{2013-03-22 16:24:28}
\pmmodified{2013-03-22 16:24:28}
\pmowner{CWoo}{3771}
\pmmodifier{CWoo}{3771}
\pmtitle{periodicity of a Markov chain}
\pmrecord{7}{38556}
\pmprivacy{1}
\pmauthor{CWoo}{3771}
\pmtype{Definition}
\pmcomment{trigger rebuild}
\pmclassification{msc}{60J10}
\pmdefines{period of a state}
\pmdefines{aperiodic state}
\pmdefines{aperiodic Markov chain}

\usepackage{amssymb,amscd}
\usepackage{amsmath}
\usepackage{amsfonts}

% used for TeXing text within eps files
%\usepackage{psfrag}
% need this for including graphics (\includegraphics)
%\usepackage{graphicx}
% for neatly defining theorems and propositions
\usepackage{amsthm}
% making logically defined graphics
%%\usepackage{xypic}
\usepackage{pst-plot}
\usepackage{psfrag}

% define commands here

\begin{document}
Let $\lbrace X_n \rbrace$ be a \PMlinkname{stationary}{StationaryProcess} Markov chain with state space $I$.  Let $P_{ij}^{n}$ be the $n$-step transition probability that the process goes from state $i$ at time $0$ to state $j$ at time $n$:
$$P_{ij}^{n}=P(X_n=j\mid X_0=i).$$

Given any state $i\in I$, define the set $$N(i):=\lbrace n\ge 1\mid P_{ii}^n>0\rbrace.$$  It is not hard to see that if $n,m\in N(i)$, then $n+m\in N(i)$.  The \emph{period} of $i$, denoted by $d(i)$, is defined as 
$$
d(i):=\begin{cases}
0&\text{if } N(i)=\varnothing,\\
\gcd (N(i))&\text{otherwise},
\end{cases}
$$ 
where $\gcd(N(i))$ is the greatest common divisor of all positive integers in $N(i)$.

A state $i\in I$ is said to be \emph{aperiodic} if $d(i)=1$.  A Markov chain is called \emph{aperiodic} if every state is aperiodic.

\textbf{Property}. If states $i,j\in I$ \PMlinkname{communicate}{MarkovChainsClassStructure}, then $d(i)=d(j)$.

\begin{proof} We will employ a common inequality involving the $n$-step transition probabilities:
$$P_{ij}^{m+n}\ge P_{ik}^m P_{kj}^n$$
for any $i,j,k\in I$ and non-negative integers $m,n$.

Suppose first that $d(i)=0$.  Since $i\leftrightarrow j$, 
$\displaystyle{ P_{ij}^n>0}$ and $P_{ji}^m>0$ for some $n,m\ge 0$.  This implies that $P_{ii}^{m+n}>0$, which forces $m+n=0$ or $m=n=0$, and hence $j=i$.

Next, assume $d(i)>0$, this means that $N(i)\ne \varnothing$.  Since $i\leftrightarrow j$, there are $r,s\ge 0$ such that $P_{ji}^r>0$ and $P_{ij}^s>0$, and so $P_{jj}^{r+s}>0$, showing $r+s\in N(j)$.  If we pick any $n\in N$, we also have $\displaystyle{ P_{jj}^{r+n+s}\ge P_{ji}^r P_{ii}^n P_{ij}^s>0 }$, or $r+s+n\in N(j)$.  But this means $d(j)$ divides both $r+s$ and $r+s+n$, and so $d(j)$ divides their difference, which is $n$.  Since $n$ is arbitrarily picked, $d(j)\mid d(i)$.  Similarly, $d(i)\mid d(j)$.  Hence $d(i)=d(j)$.
\end{proof}
%%%%%
%%%%%
\end{document}
