\documentclass[12pt]{article}
\usepackage{pmmeta}
\pmcanonicalname{StationaryIncrement}
\pmcreated{2013-03-22 15:01:25}
\pmmodified{2013-03-22 15:01:25}
\pmowner{CWoo}{3771}
\pmmodifier{CWoo}{3771}
\pmtitle{stationary increment}
\pmrecord{9}{36732}
\pmprivacy{1}
\pmauthor{CWoo}{3771}
\pmtype{Definition}
\pmcomment{trigger rebuild}
\pmclassification{msc}{60G51}
\pmdefines{stationary independent increment}

% this is the default PlanetMath preamble.  as your knowledge
% of TeX increases, you will probably want to edit this, but
% it should be fine as is for beginners.

% almost certainly you want these
\usepackage{amssymb,amscd}
\usepackage{amsmath}
\usepackage{amsfonts}

% used for TeXing text within eps files
%\usepackage{psfrag}
% need this for including graphics (\includegraphics)
%\usepackage{graphicx}
% for neatly defining theorems and propositions
%\usepackage{amsthm}
% making logically defined graphics
%%%\usepackage{xypic}

% there are many more packages, add them here as you need them

% define commands here
\begin{document}
A stochastic process $\lbrace X(t)\mid t\in T\rbrace$ of real-valued
random variables $X(t)$, where $T$ is a subset of $\mathbb{R}$, is
said have \emph{stationary increments} if the probability
distribution function for $X(s+t)-X(s)$ is fixed (the same) for all
$s\in T$ such that $s+t\in T$.  In other words, the distribution for $X(s+t)-X(s)$ is a function of ``how long'' or $t$, not ``when'' or $s$.

A stochastic process that possesses both stationary increments and
independent increments is said to have \emph{stationary independent
increments}.
%%%%%
%%%%%
\end{document}
