\documentclass[12pt]{article}
\usepackage{pmmeta}
\pmcanonicalname{ProofOfMeasurabilityOfStoppedProcesses}
\pmcreated{2013-03-22 18:39:03}
\pmmodified{2013-03-22 18:39:03}
\pmowner{gel}{22282}
\pmmodifier{gel}{22282}
\pmtitle{proof of measurability of stopped processes}
\pmrecord{4}{41393}
\pmprivacy{1}
\pmauthor{gel}{22282}
\pmtype{Proof}
\pmcomment{trigger rebuild}
\pmclassification{msc}{60G05}

\endmetadata

% almost certainly you want these
\usepackage{amssymb}
\usepackage{amsmath}
\usepackage{amsfonts}

% used for TeXing text within eps files
%\usepackage{psfrag}
% need this for including graphics (\includegraphics)
%\usepackage{graphicx}
% for neatly defining theorems and propositions
\usepackage{amsthm}
% making logically defined graphics
%%%\usepackage{xypic}

% there are many more packages, add them here as you need them

% define commands here
\newtheorem*{theorem*}{Theorem}
\newtheorem*{lemma*}{Lemma}
\newtheorem*{corollary*}{Corollary}
\newtheorem*{definition*}{Definition}
\newtheorem{theorem}{Theorem}
\newtheorem{lemma}{Lemma}
\newtheorem{corollary}{Corollary}
\newtheorem{definition}{Definition}

\begin{document}
\PMlinkescapeword{properties}

Let $(\mathcal{F}_t)_{t\in\mathbb{T}}$ be a \PMlinkname{filtration}{FiltrationOfSigmaAlgebras} on the measurable space $(\Omega,\mathcal{F})$, $\tau$ be a stopping time, and $X$ be a stochastic process. We prove the following measurability properties of the stopped process $X^\tau$.

\vspace{\baselineskip}\noindent{\bf If $X$ is jointly measurable then so is $X^\tau$.}
\par Suppose first that $X$ is a process of the form $X_t=1_A1_{\{t\ge s\}}$ for some $A\in\mathcal{F}$ and $t\in\mathbb{T}$. Then, $X^\tau_t=1_{A\cap\{\tau\ge s\}}1_{\{t\ge s\}}$ is $\mathcal{B}(\mathbb{T})\otimes\mathcal{F}$-measurable. By the functional monotone class theorem, it follows that $X^\tau$ is a bounded $\mathcal{B}(\mathbb{T})\otimes\mathcal{F}$-measurable process whenever $X$ is. By taking limits of bounded processes, this generalizes to all jointly measurable processes.

\vspace{\baselineskip}\noindent{\bf If $X$ is progressively measurable then so is $X^\tau$.}
\par For any given $t\in\mathbb{T}$, let $Y$ be the $\mathcal{B}(\mathbb{T})\otimes\mathcal{F}_t$-measurable process $Y_s= X^t_s=X_{s\wedge t}$. As $\tau\wedge t$ is $\mathcal{F}_t$-measurable, the result proven above says that $(X^\tau)^t=Y^{\tau\wedge t}$ will also be $\mathcal{B}(\mathbb{T})\otimes\mathcal{F}_t$-measurable, so $X^\tau$ is progressive.

\vspace{\baselineskip}\noindent{\bf If $X$ is optional then so is $X^\tau$.}
\par As the optional processes are generated by the right-continuous and adapted processes then it is enough to prove this result when $X$ is right-continuous and adapted. Clearly, $X^\tau$ will be right-continuous. Also, $X$ will be progressive (see measurability of stochastic processes) and, by the result proven above, it follows that $X^\tau$ is progressive and, in particular, is adapted.

\vspace{\baselineskip}\noindent{\bf If $X$ is predictable then so is $X^\tau$.}
\par By the definition of predictable processes, it is enough to prove the result in the cases where $X_t=1_A1_{\{t>s\}}$ for some $s\in\mathbb{T}$ and $A\in\mathcal{F}_s$, and $X_t=1_A1_{\{t=t_0\}}$ where $t_0$ is the minimal element of $\mathbb{T}$ and $A\in\mathcal{F}_{t_0}$.

In the first case, $X^\tau_t=1_{A\cap\{\tau>s\}}1_{\{t>s\}}$ is predictable and, in the second case, $X^\tau_t=1_{A}1_{\{t=t_0\}}+1_{A\cap\{\tau=t_0\}}1_{\{t>t_0\}}$ is predictable.

%%%%%
%%%%%
\end{document}
