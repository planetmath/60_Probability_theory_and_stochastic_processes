\documentclass[12pt]{article}
\usepackage{pmmeta}
\pmcanonicalname{SampleFunction}
\pmcreated{2013-03-22 15:21:15}
\pmmodified{2013-03-22 15:21:15}
\pmowner{gel}{22282}
\pmmodifier{gel}{22282}
\pmtitle{sample function}
\pmrecord{5}{37176}
\pmprivacy{1}
\pmauthor{gel}{22282}
\pmtype{Definition}
\pmcomment{trigger rebuild}
\pmclassification{msc}{60G17}
\pmclassification{msc}{60G05}
\pmdefines{sample path}

\endmetadata

\usepackage{amssymb,amscd}
\usepackage{amsmath}
\usepackage{amsfonts}

% used for TeXing text within eps files
%\usepackage{psfrag}
% need this for including graphics (\includegraphics)
%\usepackage{graphicx}
% for neatly defining theorems and propositions
%\usepackage{amsthm}
% making logically defined graphics
%%%\usepackage{xypic}

% define commands here
\begin{document}
Let $\lbrace X(t)\mid t\in T \rbrace$ be a stochastic process, where
$X(t)$ is a random variable on the probability space
$(\Omega,\mathcal{F},\textbf{P})$.  Writing $X(t)$ as $X(t,\omega)$,
where $t\in T$ and $\omega\in\Omega$, we see that if we fix the
sample point $\omega$, we have a function in $t$: $X_{\omega}(t)
\colon t\mapsto X(t)$. This function $X_{\omega}(t)$ of $t$ is
called a \emph{sample function}, or \emph{sample path} of the
stochastic process.
%%%%%
%%%%%
\end{document}
