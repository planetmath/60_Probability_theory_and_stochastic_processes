\documentclass[12pt]{article}
\usepackage{pmmeta}
\pmcanonicalname{CameronMartinSpace}
\pmcreated{2013-03-22 15:55:56}
\pmmodified{2013-03-22 15:55:56}
\pmowner{neldredge}{4974}
\pmmodifier{neldredge}{4974}
\pmtitle{Cameron-Martin space}
\pmrecord{6}{37941}
\pmprivacy{1}
\pmauthor{neldredge}{4974}
\pmtype{Definition}
\pmcomment{trigger rebuild}
\pmclassification{msc}{60H99}
\pmrelated{WienerMeasure}
\pmdefines{Cameron-Martin space}

% this is the default PlanetMath preamble.  as your knowledge
% of TeX increases, you will probably want to edit this, but
% it should be fine as is for beginners.

% almost certainly you want these
\usepackage{amssymb}
\usepackage{amsmath}
\usepackage{amsfonts}

% used for TeXing text within eps files
%\usepackage{psfrag}
% need this for including graphics (\includegraphics)
%\usepackage{graphicx}
% for neatly defining theorems and propositions
\usepackage{amsthm}
% making logically defined graphics
%%%\usepackage{xypic}

% there are many more packages, add them here as you need them

% define commands here
\theoremstyle{definition}
\newtheorem{definition}{Definition}
\begin{document}
\begin{definition}
Let $W(\mathbb{R}^d)$ be Wiener space.  The \emph{Cameron-Martin space} $H(\mathbb{R}^d)$ is the subspace of $W(\mathbb{R}^d)$ consisting of all paths $\omega$ such that $\omega$ is absolutely continuous and $\int_0^\infty |\omega'(s)|^2\,ds < \infty$.  (Note that if $\omega$ is absolutely continuous, then it is almost everywhere differentiable, so the integral makes sense.)
\end{definition}

This can be thought of as the set of paths with ``finite energy.''

Note that $H(\mathbb{R}^d)$ has Wiener measure $0$, since sample paths of Brownian motion are nowhere differentiable, whereas a path from $H(\mathbb{R}^d)$ is almost everywhere differentiable.

%%%%%
%%%%%
\end{document}
