\documentclass[12pt]{article}
\usepackage{pmmeta}
\pmcanonicalname{5Entanglement}
\pmcreated{2014-04-22 23:06:40}
\pmmodified{2014-04-22 23:06:40}
\pmowner{rspuzio}{6075}
\pmmodifier{rspuzio}{6075}
\pmtitle{5. Entanglement}
\pmrecord{4}{88089}
\pmprivacy{1}
\pmauthor{rspuzio}{6075}
\pmtype{Feature}

\endmetadata

% this is the default PlanetMath preamble.  as your knowledge
% of TeX increases, you will probably want to edit this, but
% it should be fine as is for beginners.

% almost certainly you want these
\usepackage{amssymb}
\usepackage{amsmath}
\usepackage{amsfonts}

% need this for including graphics (\includegraphics)
\usepackage{graphicx}
% for neatly defining theorems and propositions
\usepackage{amsthm}

% making logically defined graphics
%\usepackage{xypic}
% used for TeXing text within eps files
%\usepackage{psfrag}

% there are many more packages, add them here as you need them

% define commands here

\DeclareMathOperator{\Hom}{Hom}

\newcommand{\vecify}{{\mathcal V}}
\newcommand{\Act}{{A}}
\newcommand{\act}{{a}}
\newcommand{\Sit}{{S}}
\newcommand{\occ}{{v}}
\newcommand{\univ}{{\mathbf D}}
\newcommand{\uout}{{d_{out}}}
\newcommand{\uin}{{d_{in}}}
\newcommand{\mangle}{{\mathbf C}}

\newcommand{\psheaf}{{\mathcal F}}
\newcommand{\scat}{{\mathtt{Stoch}}}
\newcommand{\subs}{{\mathtt{Sys}}}
\newcommand{\mcat}{{\mathtt{Meas}}}
\newcommand{\eop}{{$\blacksquare$}}
\newcommand{\eod}{{${}$\\}}
\newcommand{\bra}{{\langle}}
\newcommand{\ket}{{\rangle}}

\newcommand{\cN}{{\mathcal N}}
\newcommand{\bR}{{\mathbb R}}
\newcommand{\fm}{{\mathfrak m}}
\newcommand{\cP}{{\mathcal P}}

\newtheorem{thm}{Theorem}
\newtheorem{prop}[thm]{Proposition}
\newtheorem{cor}[thm]{Corollary}

\theoremstyle{remark}
\newtheorem{eg}{Example}
\newtheorem{rem}{Remark}
\newtheorem{defn}{Definition}
\setcounter{eg}{3}
\setcounter{rem}{2}
\setcounter{defn}{9}
\setcounter{thm}{8}
\begin{document}
The proof of %Theorem~\ref{t:presheaf} 
\PMlinkname{Theorem~4}{3distributeddynamicalsystems#Thmthm4}
showed the structure presheaf has non-unique descent, 
reflecting the fact that measuring devices do not necessarily 
reduce to products of subdevices. Similarly, as we will see, 
measurements do not in general decompose into independent 
submeasurements. Entanglement, $\gamma$, quantifies how far a 
measurement diverges in bits from the product of its 
submeasurements. It turns out that $\gamma>0$ is necessary for 
a system to generate more information than the sum of its 
components: non-unique descent thus provides ``room at the top''
to build systems that perform more precise measurements 
collectively than the sum of their components. 

Entanglement has no direct relation to quantum entanglement. 
The name was chosen because of a formal resemblance between the 
two quantities, see Supplementary Information of \cite{bt:09}.

\begin{defn}
    \label{d:gamma}
	\emph{Entanglement} over partition $\cP=\{M_1\ldots M_m\}$ 
    of $src(\fm_\univ)$ is
	\begin{equation*}
		\gamma(\fm_\univ,\cP,\uout)
		=H\left[\fm_\univ^\natural\circ \uout\Big\|
        \bigotimes_{i=1}^m \pi_j\circ \fm_j^\natural\circ 
        \uout\right]
	\end{equation*}
	where $\pi_j:\vecify \Sit^\univ\rightarrow 
    \vecify \Sit^{M_j}$ and $\fm_j=\{(k,l)\in 
    \fm_\univ|k\in M_j\}$.
\end{defn}

Projecting via $\pi_j$ marginalizes onto the subspace 
$\vecify\Sit^{M_j}$. Entanglement thus compares the measurement 
performed by the entire system with submeasurements over the 
decomposition of the source occasions into partition $\cP$.

\begin{thm}
	[effective information decomposes additively when 
    entanglement is zero]
	\label{t:gamma}
	\begin{equation*}
		\gamma(\fm_\univ,\cP,\uout) = 0
		\,\,\,\,\,\implies\,\,\,\,\,
		ei(\fm_\univ,\uout)=\sum_{i=1}^m ei(\fm_j,\uout).
	\end{equation*}
\end{thm}

\noindent
Proof: Follows from the observations that (i) 
$H[p\|p_1\otimes p_2]=0$ if and only if $p=p_1\otimes p_2$; (ii)
$H[p_1\otimes p_2\| q_1\otimes q_2]=H[p_1\|q_1]+H[p_2\|q_2]$; 
and (iii) the uniform distribution on $\univ$ is a tensor of 
uniform distributions on subsystems of $\univ$.
\eop

The theorem shows the relationship between effective information
and entanglement. If a system generates more information ``than 
it should'' (meaning, more than the sum of its subsystems), then
the measurements it generates are entangled. Alternatively, only
indecomposable measurements can be more precise than the sum of 
their submeasurements.

We conclude with some detailed computations for 
$X\times Y\xrightarrow{g}Z$, %Diagram~\eqref{e:diag}
\PMlinkname{Diagram~(11)}{4measurement#id2}. 
Let $\cP=\{X|Y\}$.

\begin{thm}[entanglement and effective information for 
$g:X\times Y\rightarrow Z$]
	\label{t:g_ei}
	\begin{align*}
		\gamma(\fm_{XY},\cP,\delta_z) & = 
        \sum_{(x,y)\in g^{-1}(z)}
		\frac{1}{|g^{-1}(z)|}\log_2\frac{|g^{-1}(z)|}
        {|g^{-1}_{x\times Y}(z)|\cdot |g^{-1}_{X\times Y}(z)|} \\
		& = ei(\fm_{XY},\delta_z) - ei(\fm_{X\bullet},\delta_z) 
        - ei(\fm_{\bullet Y},\delta_z).
	\end{align*}
\end{thm}

\noindent
Proof:
The first equality follows from Propositions~%\ref{t:classmeas} 
\PMlinkname{5}{4measurement#Thmthm5}
and %\ref{t:confmeas}
\PMlinkname{6}{4measurement#Thmthm6}
\begin{equation*}
    \gamma(\fm_{XY},\cP,\delta_z)=\sum_{(x,y)\in g^{-1}(z)}
	= \sum_{(x,y)\in g^{-1}(z)}\frac{1}{|g^{-1}(z)|}\log_2
    \left[
	\frac{1}{|g^{-1}(z)|}\cdot \frac{|g^{-1}(z)|}
    {|g^{-1}_{x\times Y}(z)|}
	\frac{|g^{-1}(z)|}{|g^{-1}_{X\times Y}(z)|}\right].
\end{equation*}
From the same propositions it follows that 
$ei(\fm_{XY},\delta_z) - ei(\fm_{X\bullet},\delta_z) - 
ei(\fm_{\bullet Y},\delta_z)$ equals
\begin{gather*}
	 \log_2\frac{|X|\cdot|Y|}{|g^{-1}(x)|}-\sum_{x}
     \frac{|g^{-1}_{x\times Y}(z)|}{|g^{-1}(z)|}\log_2
	\frac{|X|\cdot|g^{-1}_{x\times Y}(z)|}{|g^{-1}(z)|}
	-\sum_y \frac{|g^{-1}_{X\times y}(z)|}{|g^{-1}(z)|}
    \log_2\frac{|Y|\cdot|g^{-1}_{X\times y}(z)|}{|g^{-1}(z)|}\\
	= \log_2 \frac{1}{g^{-1}(z)} - \sum_{(x,y)\in g^{-1}(z)}
    \frac{1}{|g^{-1}(z)|}\cdot
	\log_2\frac{|g^{-1}_{X\times y}(z)|}{|g^{-1}(z)|}\cdot
    \frac{|g^{-1}_{x\times Y}(z)|}{|g^{-1}(z)|}.
\end{gather*}
Entanglement quantifies how far the size of the pre-image of 
$g^{-1}(z)$ deviates from the sizes of its $X\times y$ and 
$x\times Y$ slices as $x$ and $y$ are varied. 
\eop

By Corollary~%\ref{t:diffmeas} 
\PMlinkname{8}{4measurement#Thmthm8}
entanglement also equals $ei(\fm_{X\bullet}\rightarrow 
\fm_{XY},\delta_z)-ei(\fm_{\bullet Y},\delta_z)$. In 
Diagram~%\eqref{e:diag} 
\PMlinkname{(11)}{4measurement#id2}
entanglement is the vertical arrow minus both arrows at the 
bottom, or the difference between opposing diagonal arrows. 
Note that the diagonal arrows from left to right are 
constructed by adding edge $v_Y\rightarrow v_Z$ to the null 
system and the subsystem $\fm_{X\bullet}=
\{v_X\rightarrow v_Z\}$ respectively.  Entanglement is the 
difference between the information generated by the diagonal 
arrows. It quantifies the difference between the information 
$\{v_Y\rightarrow v_Z\}$ generates in two different contexts.

\begin{cor}
    [characterization of disentangled set-valued functions]
	\label{t:g0z}
	Function $X\times Y\xrightarrow{g}Z$ performs a 
    disentangled measurement when outputting $z$ iff
	\begin{equation*}
		g^{-1}(z)=g^{-1}_{x\times Y}(z)\times 
        g^{-1}_{X\times y}(z)
	\end{equation*}
	for any $x,y$ such that $g(x,y)=z$.
\end{cor}

\noindent
Proof:
By Theorem~\ref{t:g_ei} entanglement is zero iff
\begin{equation*}
	|g^{-1}(z)|=|g^{-1}_{x\times Y}(z)|\cdot 
    |g^{-1}_{X\times y}(z)|
\end{equation*}
for any $x,y$ such that $g(x,y)=z$. This implies the desired 
result since $g^{-1}(z)\hookrightarrow g^{-1}_{x\times Y}(z)
\times g^{-1}_{X\times y}(z)$.
\eop

Thus, the measurement generated by $g$ is disentangled iff its 
pre-image $g^{-1}(z)$ satisfies a strong geometric 
``rectangularity'' constraint: that the pre-image decomposes 
into the product of its $x\times Y$ and $X\times y$ slices for 
all pairs of slices intersecting $g^{-1}(z)$. The 
categorizations performed within a disentangled measuring 
device have nothing to do with each other, so that the device 
is best considered as two (or more) distinct devices that 
happen to have been grouped together for the purposes of 
performing a computation.

\begin{eg}
	An XOR-gate $g:X\times Y\rightarrow Z$ outputting 0 
    generates an entangled measurement. The pre-image is 
    $g^{-1}(0)=\{00,11\}$ so the XOR-gate generates 1 bit 
    of information about occasions $v_X$ and $v_Y$. However, 
    the bit is \emph{not localizable}. The measurement 
    generates no information about occasion $v_X$ taken singly,
    since its output could have been 0 or 1 with equal 
    probability; and similarly for $v_Y$.
\end{eg}

Finally, and unsurprisingly, a function is completely 
disentangled across all its measurements iff it is a 
product of two simpler functions:

\begin{cor}
	[completely disentangled functions are products]
	\label{t:g0}
	If $X\times Y\xrightarrow{g}Z$ is surjective, then\\
	$\gamma(\fm_{XY},\cP,\delta_z)=0$ for all $z\in Z$ iff $g$ 
    decomposes into $X\times Y\xrightarrow{g_1\times g_2}
    Z_1\times Z_2=Z$ for $X\xrightarrow{g_1}Z_1$ and 
    $Y\xrightarrow{g_2}Z_2$.
\end{cor}

\noindent
Proof:
The reverse implication is trivial.
In the forward direction, note that $Z=\{g^{-1}(z)|z\in Z\}$ 
and, by Corollary \ref{t:g0z}, each pre-image has product 
structure $g^{-1}(z)=g^{-1}_{x\times Y}(Z)\times 
g^{-1}_{X\times Y}(z)$. Let $Z_1=\{g^{-1}_{X\times y}
|y\in Y\mbox{ and }z\in Z\}$ and similarly for $Z_2$. Define 
\begin{equation*}
	g_1:X\rightarrow Z_1:x\mapsto \mbox{the unique element of 
    form }g^{-1}_{X\times y}(z)
	\mbox{ containing it,}
\end{equation*}
and similarly for $g_2$.

\begin{thebibliography}{10}
%\providecommand{\bibitemdeclare}[2]{}
\providecommand{\urlprefix}{Available at }
\providecommand{\url}[1]{\texttt{#1}}
\providecommand{\href}[2]{\texttt{#2}}
\providecommand{\urlalt}[2]{\href{#1}{#2}}
\providecommand{\doi}[1]{doi:\urlalt{http://dx.doi.org/#1}{#1}}
\providecommand{\bibinfo}[2]{#2}

%\bibitemdeclare{article}{bt:09}
\bibitem{bt:09}
\bibinfo{author}{David Balduzzi} \& \bibinfo{author}{Giulio Tononi}
  (\bibinfo{year}{2009}): \emph{\bibinfo{title}{Qualia: the geometry of
  integrated information}}.
\newblock {\sl \bibinfo{journal}{PLoS Comput Biol}}
  \bibinfo{volume}{5}(\bibinfo{number}{8}), p. \bibinfo{pages}{e1000462},
  \doi{10.1371/journal.pcbi.1000462}.
  
\end{thebibliography}
\end{document}
