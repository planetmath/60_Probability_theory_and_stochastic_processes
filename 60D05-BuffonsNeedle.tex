\documentclass[12pt]{article}
\usepackage{pmmeta}
\pmcanonicalname{BuffonsNeedle}
\pmcreated{2013-03-22 16:09:28}
\pmmodified{2013-03-22 16:09:28}
\pmowner{georgiosl}{7242}
\pmmodifier{georgiosl}{7242}
\pmtitle{Buffon's needle}
\pmrecord{8}{38239}
\pmprivacy{1}
\pmauthor{georgiosl}{7242}
\pmtype{Definition}
\pmcomment{trigger rebuild}
\pmclassification{msc}{60D05}
\pmclassification{msc}{60-00}

\endmetadata

% this is the default PlanetMath preamble.  as your knowledge
% of TeX increases, you will probably want to edit this, but
% it should be fine as is for beginners.

% almost certainly you want these
\usepackage{amssymb}
\usepackage{amsmath}
\usepackage{amsfonts}

% used for TeXing text within eps files
%\usepackage{psfrag}
% need this for including graphics (\includegraphics)
%\usepackage{graphicx}
% for neatly defining theorems and propositions
%\usepackage{amsthm}
% making logically defined graphics
%%%\usepackage{xypic}

% there are many more packages, add them here as you need them

% define commands here

\begin{document}
The plane is ruled by parallel lines $ 2$ inches apart and a $ 1$-inch long needle is dropped at random on the
plane. What is the probability that it hits parallel lines?
\\\textit{Solution.}
\\The first issue is to find some appropriate probability space $(\Omega, \mathcal{F}, P)$. For this,
\begin{itemize}
\item $h=$ distance from the center of the needle to the nearest line 
\item $\theta=$ the angle that the needle makes with the horizontal ranging from $0$ to $\frac{\pi}{2}$.
\end{itemize}
These fully determine the position of the needle. Let us next take the 
\begin{enumerate}
\item The probability space is $\Omega=[0,1]\times[0,\frac{\pi}{2})$
\item The probability of an event $B$ is denoted by $P[B]$ is equal to 
$\frac{area\,\ of\,\ B}{\frac{\pi}{2}}$
\end{enumerate}
Now we denote by $A$ the event that the needle hits a horizontal line. It is easily seen that this happens
when $\sin\theta\geq \frac{h}{1/2}$. Consequently $A=\{(\theta, h)\in \Omega : h\leq \frac{\sin\theta}{2}\}$
and then we get $P[A]=\frac{2}{\pi}\int_0^\frac{\pi}{2}\frac{1}{2}\sin\theta d\theta=\frac{1}{\pi}\square$

In general case, when the length of needle is $l$ and the distance of parallel lines is $d$ provided that $l<d$, the probability we want is $\frac{2l}{\pi d}$. This is obvious just taking the $l/d$-point from one edge instead of the center of the needle.




%%%%%
%%%%%
\end{document}
