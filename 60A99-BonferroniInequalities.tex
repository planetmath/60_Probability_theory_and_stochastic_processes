\documentclass[12pt]{article}
\usepackage{pmmeta}
\pmcanonicalname{BonferroniInequalities}
\pmcreated{2013-03-22 14:30:40}
\pmmodified{2013-03-22 14:30:40}
\pmowner{kshum}{5987}
\pmmodifier{kshum}{5987}
\pmtitle{Bonferroni inequalities}
\pmrecord{9}{36049}
\pmprivacy{1}
\pmauthor{kshum}{5987}
\pmtype{Theorem}
\pmcomment{trigger rebuild}
\pmclassification{msc}{60A99}
\pmrelated{BrunsPureSieve}
\pmdefines{union bound}

% this is the default PlanetMath preamble.  as your knowledge
% of TeX increases, you will probably want to edit this, but
% it should be fine as is for beginners.

% almost certainly you want these
\usepackage{amssymb}
\usepackage{amsmath}
\usepackage{amsfonts}

% used for TeXing text within eps files
%\usepackage{psfrag}
% need this for including graphics (\includegraphics)
%\usepackage{graphicx}
% for neatly defining theorems and propositions
%\usepackage{amsthm}
% making logically defined graphics
%%%\usepackage{xypic}

% there are many more packages, add them here as you need them

% define commands here
\begin{document}
Let $E(1)$, $E(2),\ldots, E(n)$ be events in a sample space. Define
\begin{align*}
 S_1 := \sum_{i=1}^n \Pr(E(i)) \\
 S_2 := \sum_{i<j} \Pr(E(i) \cap E(j)),
\end{align*}
and for $2<k\leq n$,
\[
  S_k := \sum \Pr(E(i_1)\cap \cdots \cap E(i_k) )
\]
where the summation is taken over all ordered $k$-tuples of distinct integers.

{\bf Theorem}

For odd $k$, $1 \leq k \leq n$,
\[
\Pr(E(1)\cup\cdots\cup E(n)) \leq \sum_{j=1}^k (-1)^{j+1} S_j, 
\]
and for even $k$, $2\leq k \leq n$,
\[
\Pr(E(1)\cup\cdots\cup E(n)) \geq \sum_{j=1}^k (-1)^{j+1} S_j, 
\]


{\bf Remark} When $k=1$, the Bonferroni inequality is also known as the union bound. 
When $k=n$, we have an equality, also known as the inclusion-exclusion principle.
%%%%%
%%%%%
\end{document}
