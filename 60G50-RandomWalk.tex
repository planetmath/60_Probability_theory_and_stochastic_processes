\documentclass[12pt]{article}
\usepackage{pmmeta}
\pmcanonicalname{RandomWalk}
\pmcreated{2013-03-22 14:59:22}
\pmmodified{2013-03-22 14:59:22}
\pmowner{CWoo}{3771}
\pmmodifier{CWoo}{3771}
\pmtitle{random walk}
\pmrecord{7}{36694}
\pmprivacy{1}
\pmauthor{CWoo}{3771}
\pmtype{Definition}
\pmcomment{trigger rebuild}
\pmclassification{msc}{60G50}
\pmclassification{msc}{82B41}
\pmdefines{simple random walk}
\pmdefines{symmetric simple random walk}

\endmetadata

% this is the default PlanetMath preamble.  as your knowledge
% of TeX increases, you will probably want to edit this, but
% it should be fine as is for beginners.

% almost certainly you want these
\usepackage{amssymb,amscd}
\usepackage{amsmath}
\usepackage{amsfonts}

% used for TeXing text within eps files
%\usepackage{psfrag}
% need this for including graphics (\includegraphics)
%\usepackage{graphicx}
% for neatly defining theorems and propositions
%\usepackage{amsthm}
% making logically defined graphics
%%%\usepackage{xypic}

% there are many more packages, add them here as you need them

% define commands here
\begin{document}
\textbf{Definition}.  Let $(\Omega,\mathcal{F},\mathbf{P})$ be a
probability space and $\lbrace X_i \rbrace$ a discrete-time
stochastic process defined on $(\Omega,\mathcal{F},\mathbf{P})$,
such that the $X_i$ are iid real-valued random variables, and
$i\in\mathbb{N}$, the set of natural numbers.  The \emph{random
walk} defined on $X_i$ is the sequence of partial sums, or partial
series $$S_n\colon=\sum_{i=1}^{n}X_i.$$  If $X_i\in\lbrace -1,1
\rbrace$, then the random walk defined on $X_i$ is called a
\emph{simple random walk}.  A \emph{symmetric simple random walk} is
a simple random walk such that $\mathbf{P}(X_i=1)=1/2$.

The above defines random walks in one-dimension.  One can easily
generalize to define higher dimensional random walks, by requiring
the $X_i$ to be vector-valued (in $\mathbb{R}^n$), instead of
$\mathbb{R}$.

\textbf{Remarks}.
\begin{enumerate}
\item  Intuitively, a random walk can be viewed as movement in space
where the length and the direction of each step are random.
\item  It can be shown that, the limiting case of a random walk is a
Brownian motion (with some conditions imposed on the $X_i$ so as to
satisfy part of the defining conditions of a Brownian motion). By
limiting case we mean, loosely speaking, that the lengths of the
steps are very small, approaching 0, while the total lengths of the
walk remains a constant (so that the number of steps is very large,
approaching $\infty$).
\item  If the random variables $X_i$ defining the random walk $w_i$
are integrable with zero mean $\operatorname{E}[X_i]=0$, $S_i$ is a
martingale.
\end{enumerate}
%%%%%
%%%%%
\end{document}
