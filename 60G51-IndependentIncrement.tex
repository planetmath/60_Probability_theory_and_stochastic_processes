\documentclass[12pt]{article}
\usepackage{pmmeta}
\pmcanonicalname{IndependentIncrement}
\pmcreated{2013-03-22 15:01:22}
\pmmodified{2013-03-22 15:01:22}
\pmowner{CWoo}{3771}
\pmmodifier{CWoo}{3771}
\pmtitle{independent increment}
\pmrecord{4}{36731}
\pmprivacy{1}
\pmauthor{CWoo}{3771}
\pmtype{Definition}
\pmcomment{trigger rebuild}
\pmclassification{msc}{60G51}

\endmetadata

% this is the default PlanetMath preamble.  as your knowledge
% of TeX increases, you will probably want to edit this, but
% it should be fine as is for beginners.

% almost certainly you want these
\usepackage{amssymb,amscd}
\usepackage{amsmath}
\usepackage{amsfonts}

% used for TeXing text within eps files
%\usepackage{psfrag}
% need this for including graphics (\includegraphics)
%\usepackage{graphicx}
% for neatly defining theorems and propositions
%\usepackage{amsthm}
% making logically defined graphics
%%%\usepackage{xypic}

% there are many more packages, add them here as you need them

% define commands here
\begin{document}
A stochastic process $\lbrace X(t)\mid t\in T\rbrace$ of real-valued
random variables $X(t)$, where $T$ is linearly ordered, is said have
\emph{independent increments} if for any $a,b,c,d\in T$ such that $a<b<c<d$, $X(a)-X(b)$ and $X(c)-X(d)$ are independent random
variables.

\textbf{Remark}.  In case when $X(t)$ is monotonically non-decreasing,
as in the case of a counting process, it is customary to write
$X(b)-X(a)$ and $X(d)-X(c)$ instead of the above to emphasize the
comparison of two positive quantities (for example, the numbers of
occurrences of a certain event in some time intervals).
%%%%%
%%%%%
\end{document}
