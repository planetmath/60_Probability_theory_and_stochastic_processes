\documentclass[12pt]{article}
\usepackage{pmmeta}
\pmcanonicalname{DoobsInequalities}
\pmcreated{2013-03-22 18:39:52}
\pmmodified{2013-03-22 18:39:52}
\pmowner{gel}{22282}
\pmmodifier{gel}{22282}
\pmtitle{Doob's inequalities}
\pmrecord{7}{41407}
\pmprivacy{1}
\pmauthor{gel}{22282}
\pmtype{Theorem}
\pmcomment{trigger rebuild}
\pmclassification{msc}{60G46}
\pmclassification{msc}{60G44}
\pmclassification{msc}{60G42}
\pmsynonym{Doob's inequality}{DoobsInequalities}
%\pmkeywords{martingale}
%\pmkeywords{$L^p$-norm}
\pmrelated{KolmogorovsMartingaleInequality}
\pmdefines{Doob's maximal quadratic inequality}

% almost certainly you want these
\usepackage{amssymb}
\usepackage{amsmath}
\usepackage{amsfonts}

% used for TeXing text within eps files
%\usepackage{psfrag}
% need this for including graphics (\includegraphics)
%\usepackage{graphicx}
% for neatly defining theorems and propositions
\usepackage{amsthm}
% making logically defined graphics
%%%\usepackage{xypic}

% there are many more packages, add them here as you need them

% define commands here
\newtheorem*{theorem*}{Theorem}
\newtheorem*{lemma*}{Lemma}
\newtheorem*{corollary*}{Corollary}
\newtheorem*{definition*}{Definition}
\newtheorem{theorem}{Theorem}
\newtheorem{lemma}{Lemma}
\newtheorem{corollary}{Corollary}
\newtheorem{definition}{Definition}

\begin{document}
\PMlinkescapeword{index set}
\PMlinkescapeword{terminal value}
\PMlinkescapeword{theorem}
\PMlinkescapeword{terminal}
\PMlinkescapeword{place}
\PMlinkescapeword{bounds}
\PMlinkescapeword{terminal}

Doob's inequalities place bounds on the maximum value attained by a martingale in terms of the terminal value.
We consider a process $(X_t)_{t\in\mathbb{T}}$ defined on the filtered probability space $(\Omega,\mathcal{F},(\mathcal{F})_{t\in\mathbb{T}},\mathbb{P})$. The associated maximum process $(X^*_t)$ is
\begin{equation*}
X^*_t\equiv\sup_{s\le t}|X_s|.
\end{equation*}
The notation $\Vert \cdot\Vert_p$ for the \PMlinkname{$L^p$-norm}{LpSpace} of a random variable will be used.
In discrete-time or, more generally whenever the index set $\mathbb{T}$ is countable, then Doob's inequalities are as follows.

\begin{theorem}[Doob]\label{thm:1}
Let $(X_t)_{t\in\mathbb{T}}$ be a submartingale with countable index set $\mathbb{T}$. Then,
\begin{equation}\label{eq:1}
\mathbb{P}\left(\sup_{s\le t}X_s\ge K\right)\le K^{-1}\mathbb{E}[(X_t)_+]
\end{equation}
If $X$ is either a martingale or nonnegative submartingale then,
\begin{gather}
\label{eq:2}\mathbb{P}(X^*_t\ge K)\le K^{-1}\mathbb{E}[|X_t|],\\
\label{eq:3}\Vert X^*_t\Vert_p\le \frac{p}{p-1}\Vert X_t\Vert_p. 
\end{gather}
for every $K>0$ and $p>1$.
\end{theorem}

In particular, (\ref{eq:3}) shows that the maximum of any $L^p$-bounded martingale is itself $L^p$-bounded and, martingales $X^n$ converge to $X$ in the $L^p$-norm if and only if $(X^n-X)^*\rightarrow 0$ in the $L^p$-norm. The special case where $p=2$ gives
\begin{equation*}
\mathbb{E}[(X^*_t)^2]\le 4\mathbb{E}[X_t^2]
\end{equation*}
which is known as \emph{Doob's maximal quadratic inequality}.

Similarly, (\ref{eq:2}) shows that any $L^1$-bounded martingale is almost surely bounded and that convergence in the $L^1$-norm implies ucp convergence. Inequality (\ref{eq:1}) is also known as Kolmogorov's submartingale inequality.

Doob's inequalities are often applied to continuous-time processes, where $\mathbb{T}=\mathbb{R}_+$. In this case, $X^*_t=\sup_{s\le t}|X_s|$ is a supremum of uncountably many random variables, and need not be measurable. Instead, it is typically assumed that the processes are right-continuous, in which case, for any $t>0$ the supremum may be restricted to the countable set
\begin{equation*}
\mathbb{T}^\prime=\{s\in\mathbb{R}_+:s/t\in\mathbb{Q}\}.
\end{equation*}
Putting this into Theorem \ref{thm:1} gives the following continuous-time version of the inequalities.

\begin{theorem}[Doob]
Let $(X_t)_{t\in\mathbb{R}_+}$ be a right-continuous submartingale. Then,
\begin{equation*}
\mathbb{P}\left(\sup_{s\le t}X_s\ge K\right)\le K^{-1}\mathbb{E}[(X_t)]
\end{equation*}
for every $K>0$.
If $X$ is right-continuous and either a martingale or nonnegative submartingale then,
\begin{gather*}
\mathbb{P}(X^*_t\ge K)\le K^{-1}\mathbb{E}[|X_t|],\\
\Vert X^*_t\Vert_p\le \frac{p}{p-1}\Vert X_t\Vert_p. 
\end{gather*}
for every $K>0$ and $p>1$.
\end{theorem}

%%%%%
%%%%%
\end{document}
