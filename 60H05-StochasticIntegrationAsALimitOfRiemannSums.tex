\documentclass[12pt]{article}
\usepackage{pmmeta}
\pmcanonicalname{StochasticIntegrationAsALimitOfRiemannSums}
\pmcreated{2013-03-22 18:41:33}
\pmmodified{2013-03-22 18:41:33}
\pmowner{gel}{22282}
\pmmodifier{gel}{22282}
\pmtitle{stochastic integration as a limit of Riemann sums}
\pmrecord{4}{41452}
\pmprivacy{1}
\pmauthor{gel}{22282}
\pmtype{Theorem}
\pmcomment{trigger rebuild}
\pmclassification{msc}{60H05}
\pmclassification{msc}{60G07}
\pmclassification{msc}{60H10}
%\pmkeywords{semimartingale}
%\pmkeywords{stochastic integral}
%\pmkeywords{Riemann sum}
%\pmkeywords{partition}
\pmrelated{StochasticIntegration}

% almost certainly you want these
\usepackage{amssymb}
\usepackage{amsmath}
\usepackage{amsfonts}

% used for TeXing text within eps files
%\usepackage{psfrag}
% need this for including graphics (\includegraphics)
%\usepackage{graphicx}
% for neatly defining theorems and propositions
\usepackage{amsthm}
% making logically defined graphics
%%%\usepackage{xypic}

% there are many more packages, add them here as you need them

% define commands here
\newtheorem*{theorem*}{Theorem}
\newtheorem*{lemma*}{Lemma}
\newtheorem*{corollary*}{Corollary}
\newtheorem*{definition*}{Definition}
\newtheorem{theorem}{Theorem}
\newtheorem{lemma}{Lemma}
\newtheorem{corollary}{Corollary}
\newtheorem{definition}{Definition}

\begin{document}
\PMlinkescapeword{limit}
\PMlinkescapeword{approximations}
\PMlinkescapeword{partitions}
\PMlinkescapeword{terms}
\PMlinkescapeword{infinity}
\PMlinkescapeword{points}
\PMlinkescapeword{integral}
\PMlinkescapeword{even}
\PMlinkescapeword{necessary}
\PMlinkescapeword{left limit}

As with the \PMlinkname{Riemann}{RiemannIntegral} and Riemann-Stieltjes integrals, the stochastic integral can be calculated as a limit of approximations computed on \PMlinkname{partitions}{Partition3}, called Riemann sums.

Let $P_n$ be a sequence of partitions of $\mathbb{R}_+$,
\begin{equation*}
P_n=\left\{0=\tau^n_0\le\tau^n_1\le\cdots\uparrow\infty\right\}
\end{equation*}
where, $\tau^n_k$ can, in general, be stopping times. Suppose that the mesh $|P_n^t|=\sup_k(\tau^n_k\wedge t-\tau^n_{k-1}\wedge t)$ tends to zero \PMlinkname{in probability}{ConvergenceInProbability} as $n\rightarrow\infty$, for each time $t>0$.

The stochastic integral of a process $Y$ with respect to $X$ can be defined on each of the partitions,
\begin{equation*}
I^n_t(Y,X)\equiv\sum_k Y_{\tau^n_{k-1}}(X_{\tau^n_k\wedge t}-X_{\tau^n_{k-1}\wedge t}).
\end{equation*}
Since the times $\tau^n_k$ tend to infinity as $k\rightarrow\infty$, all but finitely many terms are zero.
Note that here, the process $Y$ is sampled at $\tau^n_{k-1}$, which are the left hand points of the intervals. For this reason, the stochastic integral is sometimes called the forward integral. Alternatively, the backward integral can be computed by sampling $Y$ at time $t_k$ and the Stratonovich integral takes the average of $Y_{t_{k-1}}$ and $Y_{t_k}$. However, these alternative integrals are less general and may not exist even when $Y$ is a continuous and adapted process.

For left-continuous integrands, the approximations do indeed converge to the stochastic integral.

\begin{theorem}
Suppose that $X$ is a semimartingale and $Y$ is an adapted, left-continuous and locally bounded process. Then,
\begin{equation*}
I^n_t(Y,X)\rightarrow\int_0^t Y\,dX
\end{equation*}
in probability as $n\rightarrow\infty$. Furthermore, this converges ucp and in the semimartingale topology.
\end{theorem}

Similarly, convergence is also obtained for cadlag integrands. However, in this case, it is necessary to use the left limit $Y_{s-}$ in the integral. The integral of $Y$ does not even exist when it is a general cadlag adapted process, as it might not be predictable.

\begin{theorem}
Suppose that $X$ is a semimartingale and $Y$ is a cadlag adapted process. Then,
\begin{equation*}
I^n_t(Y,X)\rightarrow\int_0^t Y_{s-}\,dX_s
\end{equation*}
in probability as $n\rightarrow\infty$. Furthermore, this converges ucp and in the semimartingale topology.
\end{theorem}

%%%%%
%%%%%
\end{document}
