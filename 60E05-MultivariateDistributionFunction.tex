\documentclass[12pt]{article}
\usepackage{pmmeta}
\pmcanonicalname{MultivariateDistributionFunction}
\pmcreated{2013-03-22 16:33:50}
\pmmodified{2013-03-22 16:33:50}
\pmowner{CWoo}{3771}
\pmmodifier{CWoo}{3771}
\pmtitle{multivariate distribution function}
\pmrecord{7}{38753}
\pmprivacy{1}
\pmauthor{CWoo}{3771}
\pmtype{Definition}
\pmcomment{trigger rebuild}
\pmclassification{msc}{60E05}
\pmclassification{msc}{62E10}
\pmrelated{Copula}
\pmdefines{multivariate cumulative distribution function}
\pmdefines{joint distribution function}
\pmdefines{margin}

\usepackage{amssymb,amscd}
\usepackage{amsmath}
\usepackage{amsfonts}

% used for TeXing text within eps files
%\usepackage{psfrag}
% need this for including graphics (\includegraphics)
%\usepackage{graphicx}
% for neatly defining theorems and propositions
%\usepackage{amsthm}
% making logically defined graphics
%%\usepackage{xypic}
\usepackage{pst-plot}
\usepackage{psfrag}

% define commands here

\begin{document}
A function $F:\mathbb{R}^n\to [0,1]$ is said to be a \emph{multivariate distribution function} if
\begin{enumerate}
\item $F$ is non-decreasing in each of its arguments; i.e., for any $1\le i\le n$, the function $G_i:\mathbb{R}\to [0,1]$ given by $G_i(x):=F(a_1,\ldots,a_{i-1},x,a_{i+1},\ldots,a_n)$ is non-decreasing for any set of $a_j\in \mathbb{R}$ such that $j\ne i$.
\item $G_i(-\infty)=0$, where $G_i$ is defined as above; i.e., the limit of $G_i$ as $x\to -\infty$ is $0$
\item $F(\infty,\ldots,\infty)=1$; i.e. the limit of $F$ as each of its arguments approaches infinity, is 1.
\end{enumerate}

Generally, right-continuty of $F$ in each of its arguments is added as one of the conditions, but it is not assumed here.

If, in the second condition above, we set $a_j=\infty$ for $j\ne i$, then $G_i(x)$ is called a (one-dimensional) \emph{margin} of $F$.  Similarly, one defines an $m$-dimensional ($m<n$) \emph{margin} of $F$ by setting $n-m$ of the arguments in $F$ to $\infty$.  For each $m<n$, there are $\binom{n}{m}$ $m$-dimensional margins of $F$.  Each $m$-dimensional margin of a multivariate distribution function is itself a multivariate distribution function.  A one-dimensional margin is a distribution function.

Multivariate distribution functions are typically found in probability theory, and especially in statistics.  An example of a commonly used multivariate distribution function is the multivariate Gaussian distribution function.  In $\mathbb{R}^2$, the standard bivariate Gaussian distribution function (with zero mean vector, and the identity matrix as its covariance matrix) is given by
$$F(x,y)=\frac{1}{2\pi}\int_{-\infty}^x \int_{-\infty}^y \operatorname{exp}\big({-\frac{s^2+t^2}{2}}\big) ds dt$$

B. Schweizer and A. Sklar have generalized the above definition to include a wider class of functions.  The generalization has to do with the weakening of the coordinate-wise non-decreasing condition (first condition above).  The attempt here is to study a class of functions that can be used as models for distributions of distances between points in a ``probabilistic metric space''.

\begin{thebibliography}{8}
\bibitem{bs as} B. Schweizer and A. Sklar, {\em Probabilistic Metric Spaces}, Dover Publications, (2005).
\end{thebibliography}
%%%%%
%%%%%
\end{document}
