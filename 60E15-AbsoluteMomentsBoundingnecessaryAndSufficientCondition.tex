\documentclass[12pt]{article}
\usepackage{pmmeta}
\pmcanonicalname{AbsoluteMomentsBoundingnecessaryAndSufficientCondition}
\pmcreated{2013-03-22 16:13:58}
\pmmodified{2013-03-22 16:13:58}
\pmowner{Andrea Ambrosio}{7332}
\pmmodifier{Andrea Ambrosio}{7332}
\pmtitle{absolute moments bounding (necessary and sufficient condition)}
\pmrecord{5}{38333}
\pmprivacy{1}
\pmauthor{Andrea Ambrosio}{7332}
\pmtype{Theorem}
\pmcomment{trigger rebuild}
\pmclassification{msc}{60E15}

% this is the default PlanetMath preamble.  as your knowledge
% of TeX increases, you will probably want to edit this, but
% it should be fine as is for beginners.

% almost certainly you want these
\usepackage{amssymb}
\usepackage{amsmath}
\usepackage{amsfonts}

% used for TeXing text within eps files
%\usepackage{psfrag}
% need this for including graphics (\includegraphics)
%\usepackage{graphicx}
% for neatly defining theorems and propositions
\usepackage{amsthm}
% making logically defined graphics
%%%\usepackage{xypic}

% there are many more packages, add them here as you need them

% define commands here

\begin{document}
Let $X$ be a random variable; then%
\[
E[\left\vert X\right\vert ^{k}]\leq M^{k}\text{ \ \ \ \ \ \ }\forall k\geq
1,k\in \mathbf{N}
\]

\bigskip if and only if,$\forall i\geq 0,i\in \mathbf{N}$ 
\[
E\left[ \left\vert X\right\vert ^{k}\right] \leq E\left[ \left\vert
X\right\vert ^{i}\right] M^{k-i}\text{ \ \ \ \ \ \ \ }\forall k\geq i,k\in \mathbf{N}
\]

\bigskip 

\begin{proof}


a) $(E\left[ \left\vert X\right\vert ^{k}\right] \leq E\left[
\left\vert X\right\vert ^{i}\right] M^{k-i}$ \ $\Longrightarrow $ \ $%
E[\left\vert X\right\vert ^{k}]\leq M^{k})$

It's enough to take $i=0$ and the thesis follows easily.

\bigskip 

b)  $(E[\left\vert X\right\vert ^{k}]\leq M^{k}$ $\Longrightarrow E\left[
\left\vert X\right\vert ^{k}\right] \leq E\left[ \left\vert X\right\vert ^{i}%
\right] M^{k-i})$

Let $1\leq i\leq k$ (the case $i=0$ is trivial). Then, using Cauchy-Schwarz
inequality $N$ times, one has:%
\begin{eqnarray*}
E[\left\vert X\right\vert ^{k}] &=&E\left[ \left\vert X\right\vert ^{\frac{i%
}{2}}\left\vert X\right\vert ^{k-\frac{i}{2}}\right]  \\
&\leq &E\left[ \left\vert X\right\vert ^{i}\right] ^{\frac{1}{2}}E\left[
\left\vert X\right\vert ^{2k-i}\right] ^{\frac{1}{2}} \\
&=&E\left[ \left\vert X\right\vert ^{i}\right] ^{\frac{1}{2}}E\left[
\left\vert X\right\vert ^{\frac{i}{2}}\left\vert X\right\vert ^{2k-\frac{3}{2%
}i}\right] ^{\frac{1}{2}} \\
&\leq &E\left[ \left\vert X\right\vert ^{i}\right] ^{\left( \frac{1}{2}+%
\frac{1}{4}\right) }E\left[ \left\vert X\right\vert ^{4k-3i}\right] ^{\frac{1%
}{4}} \\
&\leq &E\left[ \left\vert X\right\vert ^{i}\right] ^{\left( \frac{1}{2}+%
\frac{1}{4}+\frac{1}{8}\right) }E\left[ \left\vert X\right\vert ^{\left(
8k-7i\right) }\right] ^{\frac{1}{8}} \\
&&... \\
&\leq &E\left[ \left\vert X\right\vert ^{i}\right] ^{\left( \sum_{m=1}^{N}%
\frac{1}{2^{m}}\right) }E\left[ \left\vert X\right\vert ^{2^{N}k-\left(
2^{N}-1\right) i}\right] ^{\frac{1}{2^{N}}} \\
&=&E\left[ \left\vert X\right\vert ^{i}\right] ^{\left( 1-\frac{1}{2^{N}}%
\right) }E\left[ \left\vert X\right\vert ^{2^{N}\left( k-i\right) +i}\right]
^{\frac{1}{2^{N}}} \\
&\leq &E\left[ \left\vert X\right\vert ^{i}\right] ^{\left( 1-\frac{1}{2^{N}}%
\right) }M^{\left( k-i\right) +\frac{i}{2^{N}}},
\end{eqnarray*}

and since this must hold for any $N$, we obtain
\[
E[\left\vert X\right\vert ^{k}]\leq E\left[ \left\vert X\right\vert ^{i}%
\right] M^{k-i}
\]
\end{proof}
%%%%%
%%%%%
\end{document}
