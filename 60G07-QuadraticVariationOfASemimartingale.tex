\documentclass[12pt]{article}
\usepackage{pmmeta}
\pmcanonicalname{QuadraticVariationOfASemimartingale}
\pmcreated{2013-03-22 18:41:21}
\pmmodified{2013-03-22 18:41:21}
\pmowner{gel}{22282}
\pmmodifier{gel}{22282}
\pmtitle{quadratic variation of a semimartingale}
\pmrecord{4}{41441}
\pmprivacy{1}
\pmauthor{gel}{22282}
\pmtype{Theorem}
\pmcomment{trigger rebuild}
\pmclassification{msc}{60G07}
\pmclassification{msc}{60G48}
\pmclassification{msc}{60H05}
%\pmkeywords{semimartingale}
%\pmkeywords{quadratic variation}
%\pmkeywords{quadratic covariation}
\pmrelated{QuadraticVariation}

\endmetadata

% almost certainly you want these
\usepackage{amssymb}
\usepackage{amsmath}
\usepackage{amsfonts}

% used for TeXing text within eps files
%\usepackage{psfrag}
% need this for including graphics (\includegraphics)
%\usepackage{graphicx}
% for neatly defining theorems and propositions
\usepackage{amsthm}
% making logically defined graphics
%%%\usepackage{xypic}

% there are many more packages, add them here as you need them

% define commands here
\newtheorem*{theorem*}{Theorem}
\newtheorem*{lemma*}{Lemma}
\newtheorem*{corollary*}{Corollary}
\newtheorem*{definition*}{Definition}
\newtheorem{theorem}{Theorem}
\newtheorem{lemma}{Lemma}
\newtheorem{corollary}{Corollary}
\newtheorem{definition}{Definition}

\begin{document}
\PMlinkescapeword{formula}
\PMlinkescapeword{consequence}
\PMlinkescapeword{ucp convergence}
\PMlinkescapeword{satisfy}
\PMlinkescapeword{sum}
\PMlinkescapeword{squares}

Given any semimartingale $X$, its quadratic variation $[X]$ exists and, for any two semimartingales $X,Y$, their quadratic covariation $[X,Y]$ exists. This is a consequence of the existence of the stochastic integral, and the covariation can be expressed by the integration by parts formula
\begin{equation*}
[X,Y]_t=X_tY_t-X_0Y_0-\int_0^tX_{s-}\,dY_s-\int_0^tY_{s-}\,dX_s.
\end{equation*}
Furthermore, suppose that $P_n$ is a sequence of \PMlinkname{partitions}{Partition3} of $\mathbb{R}_+$,
\begin{equation*}
P_n=\left\{0=\tau^n_0\le\tau^n_1\le\cdots\uparrow\infty\right\}
\end{equation*}
where, $\tau^n_k$ can, in general, be stopping times. Suppose that the mesh $|P_n^t|=\sup_k(\tau^n_k\wedge t-\tau^n_{k-1}\wedge t)$ tends to zero in probability as $n\rightarrow\infty$, for each time $t>0$.
Then, the approximations $[X,Y]^{P_n}$ to the quadratic covariation \PMlinkname{converge ucp}{UcpConvergence} to $[X,Y]$ and, convergence also holds in the semimartingale topology.

A consequence of ucp convergence is that the jumps of the quadratic variation and covariation satisfy
\begin{equation*}
\Delta[X]=(\Delta X)^2,\ \Delta[X,Y]=\Delta X\Delta Y
\end{equation*}
at all times.
In particular, $[X,Y]$ is continuous whenever $X$ or $Y$ is continuous.
As quadratic variations are increasing processes, this shows that the sum of the squares of the jumps of a semimartingale is finite over any bounded interval
\begin{equation*}
\sum_{s\le t}(\Delta X_s)^2 \le [X]_t <\infty.
\end{equation*}

Given any two semimartingales $X$,$Y$, the polarization identity $[X,Y]=([X+Y]-[X-Y])/4$ expresses the covariation as a difference of increasing processes and, therefore is of \PMlinkname{finite variation}{FiniteVariationProcess}, So, the continuous part of the covariation
\begin{equation*}
[X,Y]^c_t\equiv [X,Y]_t-\sum_{s\le t}\Delta X_s\Delta Y_s
\end{equation*}
is well defined and continuous.

%%%%%
%%%%%
\end{document}
