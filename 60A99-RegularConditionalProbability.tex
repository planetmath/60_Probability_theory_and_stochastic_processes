\documentclass[12pt]{article}
\usepackage{pmmeta}
\pmcanonicalname{RegularConditionalProbability}
\pmcreated{2013-03-22 16:25:24}
\pmmodified{2013-03-22 16:25:24}
\pmowner{CWoo}{3771}
\pmmodifier{CWoo}{3771}
\pmtitle{regular conditional probability}
\pmrecord{8}{38574}
\pmprivacy{1}
\pmauthor{CWoo}{3771}
\pmtype{Definition}
\pmcomment{trigger rebuild}
\pmclassification{msc}{60A99}

\endmetadata

\usepackage{amssymb,amscd}
\usepackage{amsmath}
\usepackage{amsfonts}

% used for TeXing text within eps files
%\usepackage{psfrag}
% need this for including graphics (\includegraphics)
%\usepackage{graphicx}
% for neatly defining theorems and propositions
%\usepackage{amsthm}
% making logically defined graphics
%%\usepackage{xypic}
\usepackage{pst-plot}
\usepackage{psfrag}

% define commands here

\begin{document}
\subsubsection*{Introduction}
Suppose $(\Omega, \mathcal{F}, P)$ is a probability space and $B\in\mathcal{F}$ be an event with $P(B)>0$.  It is easy to see that $P_B:\mathcal{F}\to [0,1]$ defined by $$P_B(A):=P(A |B),$$ the conditional probability of event $A$ given $B$, is a probability measure defined on $\mathcal{F}$, since:
\begin{enumerate}
\item $P_B$ is clearly non-negative;
\item $P_B(\Omega)=\displaystyle{\frac{P(\Omega\cap B)}{P(B)}=\frac{P(B)}{P(B)}=1}$;
\item $P_B$ is countably additive: for if $\lbrace A_1,A_2,\ldots\rbrace$ is a countable collection of pairwise disjoint events in $\mathcal{F}$, then $$P_B(\bigcup_{i=1}^{\infty} A_i)=\frac{P\big(B\cap (\bigcup A_i)\big)}{P(B)} =\frac{P\big(\bigcup (B\cap A_i)\big)}{P(B)} = \frac{\sum P(B\cap A_i)}{P(B)} = \sum_{i=1}^{\infty}P_B(A_i),$$ as $\lbrace B\cap A_1,B\cap A_2,\ldots\rbrace$ is a collection of pairwise disjoint events also.
\end{enumerate}
\subsubsection*{Regular Conditional Probability}
Can we extend the definition above to $P_{\mathcal{G}}$, where $\mathcal{G}$ is a sub sigma algebra of $\mathcal{F}$ instead of an event?  First, we need to be careful what we mean by $P_{\mathcal{G}}$, since, given any event $A\in\mathcal{F}$, $P(A|\mathcal{G})$ is not a real number.  And strictly speaking, it is not even a random variable, but an equivalence class of random variables (each pair differing by a null event in $\mathcal{G}$).

With this in mind, we start with a probability measure $P$ defined on $\mathcal{F}$ and a sub sigma algebra $\mathcal{G}$ of $\mathcal{F}$.  A function $P_{\mathcal{G}}:\mathcal{G}\times\Omega\to [0,1]$ is a called a \emph{regular conditional probability} if it has the following properties:
\begin{enumerate}
\item for each event $A\in\mathcal{G}$, $P_{\mathcal{G}}(A,\cdot):\Omega\to [0,1]$ is a \PMlinkname{conditional probability}{ProbabilityConditioningOnASigmaAlgebra} (as a random variable) of $A$ given $\mathcal{G}$; that is, 
\begin{enumerate}
\item $P_{\mathcal{G}}(A,\cdot)$ is \PMlinkname{$\mathcal{G}$-measurable}{MathcalFMeasurableFunction} and 
\item for every $B\in\mathcal{G}$, we have $\displaystyle \int_B P_{\mathcal{G}}(A,\cdot) dP =P(A\cap B).$
\end{enumerate}
\item for every outcome $\omega\in \Omega$, $P_{\mathcal{G}}(\cdot,\omega): \mathcal{G}\to [0,1]$ is a probability measure.
\end{enumerate}

There are probability spaces where no regular conditional probabilities can be defined.  However, when a regular conditional probability function does exist on a space $\Omega$, then by condition 2 of the definition, we can define a ``conditional'' probability measure on $\Omega$ for each outcome in the sense of the first two paragraphs.
%%%%%
%%%%%
\end{document}
