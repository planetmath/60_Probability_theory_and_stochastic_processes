\documentclass[12pt]{article}
\usepackage{pmmeta}
\pmcanonicalname{MartingaleConvergenceTheorem}
\pmcreated{2013-03-22 18:33:47}
\pmmodified{2013-03-22 18:33:47}
\pmowner{gel}{22282}
\pmmodifier{gel}{22282}
\pmtitle{martingale convergence theorem}
\pmrecord{5}{41286}
\pmprivacy{1}
\pmauthor{gel}{22282}
\pmtype{Theorem}
\pmcomment{trigger rebuild}
\pmclassification{msc}{60G46}
\pmclassification{msc}{60G44}
\pmclassification{msc}{60G42}
\pmclassification{msc}{60F15}
%\pmkeywords{martingale}
%\pmkeywords{submartingale}
%\pmkeywords{supermartingale}
\pmrelated{Martingale}
\pmrelated{MartingaleProofOfKolmogorovsStrongLawForSquareIntegrableVariables}
\pmrelated{MartingaleProofOfTheRadonNikodymTheorem}
\pmrelated{UpcrossingsAndDowncrossings}

\endmetadata

% this is the default PlanetMath preamble.  as your knowledge
% of TeX increases, you will probably want to edit this, but
% it should be fine as is for beginners.

% almost certainly you want these
\usepackage{amssymb}
\usepackage{amsmath}
\usepackage{amsfonts}

% used for TeXing text within eps files
%\usepackage{psfrag}
% need this for including graphics (\includegraphics)
%\usepackage{graphicx}
% for neatly defining theorems and propositions
\usepackage{amsthm}
% making logically defined graphics
%%%\usepackage{xypic}

% there are many more packages, add them here as you need them

% define commands here
\newtheorem*{theorem*}{Theorem}
\newtheorem*{lemma*}{Lemma}
\newtheorem*{corollary*}{Corollary}
\newtheorem{theorem}{Theorem}
\newtheorem{lemma}{Lemma}
\newtheorem{corollary}{Corollary}


\begin{document}
There are several convergence theorems for martingales, which follow from Doob's upcrossing lemma. The following says that any $L^1$-bounded martingale $X_n$ in discrete time converges almost surely.
Note that almost-sure convergence (i.e. convergence with probability one) is quite strong, implying the weaker property of convergence in probability. Here, a martingale $(X_n)_{n\in\mathbb{N}}$ is understood to be defined with respect to a probability space $(\Omega,\mathcal{F},\mathbb{P})$ and filtration $(\mathcal{F}_n)_{n\in\mathbb{N}}$.

\begin{theorem*}[Doob's Forward Convergence Theorem]
Let $(X_n)_{n\in\mathbb{N}}$ be a martingale (or submartingale, or supermartingale) such that $\mathbb{E}[|X_n|]$ is bounded over all $n\in\mathbb{N}$. Then, with probability one, the limit $X_\infty=\lim_{n\rightarrow\infty}X_n$ exists and is finite.
\end{theorem*}

The condition that $X_n$ is $L^1$-bounded is automatically satisfied in many cases. In particular, if $X$ is a non-negative supermartingale then $\mathbb{E}[|X_n|]=\mathbb{E}[X_n]\le\mathbb{E}[X_1]$ for all $n\ge 1$, so $\mathbb{E}[|X_n|]$ is bounded, giving the following corollary.

\begin{corollary*}
Let $(X_n)_{n\in\mathbb{N}}$ be a non-negative martingale (or supermartingale). Then, with probability one, the limit $X_\infty=\lim_{n\rightarrow\infty}X_n$ exists and is finite.
\end{corollary*}

As an example application of the martingale convergence theorem, it is easy to show that a standard random walk started started at $0$ will visit every level with probability one.

\begin{corollary*}
Let $(X_n)_{n\in\mathbb{N}}$ be a standard random walk. That is, $X_1=0$ and
\begin{equation*}
\mathbb{P}(X_{n+1}=X_n+1\mid \mathcal{F}_n)=\mathbb{P}(X_{n+1}=X_n-1\mid\mathcal{F}_n) = 1/2.
\end{equation*}
Then, for every integer $a$, with probability one $X_n=a$ for some $n$.
\end{corollary*}
\begin{proof}
Without loss of generality, suppose that $a\le 0$. Let $T:\Omega\rightarrow\mathbb{N}\cup\{\infty\}$ be the first time $n$ for which $X_n=a$. It is easy to see that the stopped process $X^T_n$ defined by $X^T_n=X_{\min(n,T)}$ is a martingale and $X^T-a$ is non-negative. Therefore, by the martingale convergence theorem, the limit $X^T_\infty=\lim_{n\rightarrow\infty}X^T_n$ exists and is finite (almost surely). In particular, $|X^T_{n+1}-X^T_n|$ converges to $0$ and must be less than $1$ for large $n$. However, $|X^T_{n+1}-X^T_n|=1$ whenever $n<T$, so we have $T<\infty$ and therefore $X_n=a$ for some $n$.
\end{proof}

%%%%%
%%%%%
\end{document}
