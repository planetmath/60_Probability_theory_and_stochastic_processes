\documentclass[12pt]{article}
\usepackage{pmmeta}
\pmcanonicalname{PoissonProcess}
\pmcreated{2013-03-22 15:01:29}
\pmmodified{2013-03-22 15:01:29}
\pmowner{CWoo}{3771}
\pmmodifier{CWoo}{3771}
\pmtitle{Poisson process}
\pmrecord{8}{36733}
\pmprivacy{1}
\pmauthor{CWoo}{3771}
\pmtype{Definition}
\pmcomment{trigger rebuild}
\pmclassification{msc}{60G51}
\pmsynonym{homogeneous Poisson process}{PoissonProcess}
\pmdefines{simple Poisson process}
\pmdefines{intensity}

\endmetadata

% this is the default PlanetMath preamble.  as your knowledge
% of TeX increases, you will probably want to edit this, but
% it should be fine as is for beginners.

% almost certainly you want these
\usepackage{amssymb,amscd}
\usepackage{amsmath}
\usepackage{amsfonts}

% used for TeXing text within eps files
%\usepackage{psfrag}
% need this for including graphics (\includegraphics)
%\usepackage{graphicx}
% for neatly defining theorems and propositions
%\usepackage{amsthm}
% making logically defined graphics
%%%\usepackage{xypic}

% there are many more packages, add them here as you need them

% define commands here
\begin{document}
\PMlinkescapeword{simple}

A counting process $\lbrace X(t)\mid
t\in\mathbb{R}^{+}\cup\lbrace0\rbrace\rbrace$ is called a
\emph{simple Poisson}, or simply a \emph{Poisson process} with
parameter $\lambda$, also known as the \emph{intensity}, if
\begin{enumerate}
\item $X(0)=0$,
\item $\lbrace X(t)\rbrace$ has stationary independent increments,
\item $P(X(t)=1)=\lambda t+o(t)$,
\item $P(X(t)>1)=o(t)$,
\end{enumerate}
where $o(t)$ is the O notation.

\textbf{Remarks}.
\begin{itemize}
\item The intensity $\lambda$ is assumed to be a constant in terms of $t$.
\item Condition 3 above says that the \emph{rate} in which the an event occurs once in time interval $t$, as $t$ approaches 0, is $\lambda$. Condition 4 says that the event occurs more than once is very unlikely (the rate approaches zero as the time interval shrinks to zero).
\item It can be shown that $X(t)$ has a Poisson distribution (hence the name of the stochastic process) with parameter $\lambda t$: $$P(X(t)=n)=e^{-\lambda t}\frac{{(\lambda t)}^n}{n!}.$$
\item Therefore, $\operatorname{E}[X(t)]=\lambda t$.
\end{itemize}
%%%%%
%%%%%
\end{document}
