\documentclass[12pt]{article}
\usepackage{pmmeta}
\pmcanonicalname{ContinuousDensityFunction}
\pmcreated{2013-03-22 11:53:16}
\pmmodified{2013-03-22 11:53:16}
\pmowner{mathcam}{2727}
\pmmodifier{mathcam}{2727}
\pmtitle{continuous density function}
\pmrecord{9}{30496}
\pmprivacy{1}
\pmauthor{mathcam}{2727}
\pmtype{Definition}
\pmcomment{trigger rebuild}
\pmclassification{msc}{60-00}
\pmclassification{msc}{81-00}
\pmclassification{msc}{18-00}
\pmsynonym{mass function}{ContinuousDensityFunction}

\usepackage{amssymb}
%\usepackage{amsmath}
%\usepackage{amsfonts}
%\usepackage{graphicx}
%%%%%\usepackage{xypic}
\begin{document}
\PMlinkescapeword{continuous}
\PMlinkescapeword{information}
\PMlinkescapeword{satisfy}

Let $X$ be a continuous random variable. The function $f_X\colon\mathbb{R} \to [0,1]$ defined as $ f_X(x) = \frac{\partial F_X}{\partial x} $, where $ F_X(x)$ is the \PMlinkname{cumulative distribution function}{CumulativeDistributionFunction} of $X$, is called the \emph{continuous density function} of $X$.
Please note that if $X$ is a continuous random variable, then $f_X(x)$ does not equal $P[X=x]$; for more information read the article on cumulative distribution functions.

Analogously to the discrete case, this function must satisfy:

\begin{enumerate}
\item $f_X(x) \geq 0$ for all $x$
\item $\int_{x}^{} {f_X(x) dx} = 1$
\end{enumerate}
%%%%%
%%%%%
%%%%%
%%%%%
\end{document}
