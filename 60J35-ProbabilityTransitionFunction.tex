\documentclass[12pt]{article}
\usepackage{pmmeta}
\pmcanonicalname{ProbabilityTransitionFunction}
\pmcreated{2013-03-22 16:12:37}
\pmmodified{2013-03-22 16:12:37}
\pmowner{mcarlisle}{7591}
\pmmodifier{mcarlisle}{7591}
\pmtitle{probability transition function}
\pmrecord{8}{38306}
\pmprivacy{1}
\pmauthor{mcarlisle}{7591}
\pmtype{Definition}
\pmcomment{trigger rebuild}
\pmclassification{msc}{60J35}
%\pmkeywords{random stochastic process transition function semigroup probability}
\pmdefines{probability transition function}
\pmdefines{homogeneous probability transition function}
\pmdefines{Chapman-Kolmogorov equation}

% this is the default PlanetMath preamble.  as your knowledge
% of TeX increases, you will probably want to edit this, but
% it should be fine as is for beginners.

% almost certainly you want these
\usepackage{amssymb}
\usepackage{amsmath}
\usepackage{amsfonts}

% used for TeXing text within eps files
%\usepackage{psfrag}
% need this for including graphics (\includegraphics)
%\usepackage{graphicx}
% for neatly defining theorems and propositions
%\usepackage{amsthm}
% making logically defined graphics
%%%\usepackage{xypic}

% there are many more packages, add them here as you need them

% define commands here

\begin{document}
A \emph{probability transition function} (p.t.f., or just t.f. in context) on a measurable space $(\Omega, \mathcal{F})$ is a family $P_{s,t}$, $0 \leq s < t$ of transition probabilities on $(\Omega, \mathcal{F})$ such that for every three real numbers $s < t < v$, the family \PMlinkescapetext{satisfies} the \emph{Chapman-Kolmogorov equation} 
\[\int P_{s,t}(x, dy)P_{t,v}(y, A) = P_{s,v}(x, A)\]
for every $x \in \Omega$ and $A \in \mathcal{F}$.  The t.f. is said to be  \PMlinkescapetext{\emph{homogeneous}} if $P_{s,t}$ depends on $s$ and $t$ only through their \PMlinkescapetext{difference} $t-s$.  In this case, we write $P_{t,0} = P_t$ and the family $\{P_t, t \geq 0\}$ is a semigroup, and the Chapman-Kolmogorov equation reads 
\[P_{t+s}(x, A) = \int P_s(x, dy) P_t(y, A).\]

\begin{thebibliography}{1}
\bibitem{RY} D. Revuz \& M. Yor, \emph{Continuous Martingales and Brownian Motion}, Third Edition Corrected. Volume 293, Grundlehren der mathematischen Wissenschaften. Springer, Berlin, 2005.
\end{thebibliography}
%%%%%
%%%%%
\end{document}
