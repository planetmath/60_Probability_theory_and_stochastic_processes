\documentclass[12pt]{article}
\usepackage{pmmeta}
\pmcanonicalname{BichtelerDellacherieTheorem}
\pmcreated{2013-03-22 18:36:48}
\pmmodified{2013-03-22 18:36:48}
\pmowner{gel}{22282}
\pmmodifier{gel}{22282}
\pmtitle{Bichteler-Dellacherie theorem}
\pmrecord{8}{41347}
\pmprivacy{1}
\pmauthor{gel}{22282}
\pmtype{Theorem}
\pmcomment{trigger rebuild}
\pmclassification{msc}{60G48}
\pmclassification{msc}{60H05}
\pmclassification{msc}{60G07}
\pmsynonym{Dellacherie-Meyer-Mokobodzky theorem}{BichtelerDellacherieTheorem}
%\pmkeywords{semimartingale}
%\pmkeywords{local martingale}
%\pmkeywords{finite variation process}
%\pmkeywords{simple predictable process}

\endmetadata

% almost certainly you want these
\usepackage{amssymb}
\usepackage{amsmath}
\usepackage{amsfonts}

% used for TeXing text within eps files
%\usepackage{psfrag}
% need this for including graphics (\includegraphics)
%\usepackage{graphicx}
% for neatly defining theorems and propositions
\usepackage{amsthm}
% making logically defined graphics
%%%\usepackage{xypic}

% there are many more packages, add them here as you need them

% define commands here

\newtheorem*{theorem*}{Theorem}
\newtheorem*{lemma*}{Lemma}
\newtheorem*{corollary*}{Corollary}
\newtheorem*{definition*}{Definition}
\newtheorem{theorem}{Theorem}
\newtheorem{lemma}{Lemma}
\newtheorem{corollary}{Corollary}
\newtheorem{definition}{Definition}

\begin{document}
\PMlinkescapeword{calculus}
\PMlinkescapeword{states}
\PMlinkescapeword{equivalence}
\PMlinkescapeword{equivalent}
\PMlinkescapeword{theorem}
\PMlinkescapeword{theory}
\PMlinkescapeword{sums}
\PMlinkescapeword{satisfy}
\PMlinkescapeword{bounded}
\PMlinkescapeword{decomposition}
\PMlinkescapeword{property}

The Bichteler-Dellacherie theorem is an important result in stochastic calculus, and states the equivalence of two very different definitions of semimartingales.
The result also goes under other names, such as the \emph{Dellacherie-Meyer-Mokobodzky theorem}.
Prior to its discovery, a theory of stochastic integration had been developed for local martingales. As standard Lebesgue-Stieltjes integration can be applied to finite variation processes, this allowed an integral to be defined with respect to sums of local martingales and finite variation processes, known as a semimartingales. The Bichteler-Dellacherie theorem then shows that, as long as we require stochastic integration to satisfy bounded convergence, then semimartingales are actually the most general objects which can be used.

We consider a real valued stochastic process $X$ adapted to a filtered probability space $(\Omega,\mathcal{F},(\mathcal{F}_t)_{t\in\mathbb{R}_+},\mathbb{P})$. Then, the integral $\int_0^t\xi\,dX$ can be written out explicitly for any simple predictable process $\xi$.

\begin{theorem*}[Bichteler-Dellacherie]
Let $X$ be a cadlag adapted stochastic process. Then, the following are equivalent.
\begin{enumerate}
\item\label{bounded} For every $t>0$, the set
\begin{equation*}
\left\{\int_0^t\xi\,dX:|\xi|\le 1\textrm{ is simple predictable}\right\}
\end{equation*}
is bounded in probability.
\item\label{decomp} A decomposition $X=M+V$ exists, where $M$ is a local martingale and $V$ is a finite variation process.
\item\label{strong decomp} A decomposition $X=M+V$ exists, where $M$ is locally a uniformly bounded martingale and $V$ is a finite variation process.
\end{enumerate}
\end{theorem*}

Condition \ref{bounded} is equivalent to stating that if $\xi^n$ is a sequence of simple predictable processes converging uniformly to zero, then the integrals $\int_0^t\xi^n\,dX$ tend to zero in probability as $n\rightarrow\infty$, which is a weak form of bounded convergence for stochastic integration.

Conditions \ref{bounded} and \ref{decomp} are the two definitions often used for the process $X$ to be a semimartingale.
However, condition \ref{strong decomp} gives a stronger decomposition which is often more useful in practise. The property that $M$ is locally a uniformly bounded martingale means that there exists a sequence of stopping times $\tau_n$, almost surely increasing to infinity, such that the stopped processes $M^{\tau_n}$ are uniformly bounded martingales.

\begin{thebibliography}{9}
\bibitem{protter}
Philip E. Protter, \emph{Stochastic integration and differential equations.} Second edition. Applications of Mathematics, 21. Stochastic Modelling and Applied Probability. Springer-Verlag, 2004.
\end{thebibliography}

%%%%%
%%%%%
\end{document}
