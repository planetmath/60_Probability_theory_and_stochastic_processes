\documentclass[12pt]{article}
\usepackage{pmmeta}
\pmcanonicalname{InfinitelyDivisibleRandomVariable}
\pmcreated{2013-03-22 16:25:58}
\pmmodified{2013-03-22 16:25:58}
\pmowner{CWoo}{3771}
\pmmodifier{CWoo}{3771}
\pmtitle{infinitely divisible random variable}
\pmrecord{8}{38585}
\pmprivacy{1}
\pmauthor{CWoo}{3771}
\pmtype{Definition}
\pmcomment{trigger rebuild}
\pmclassification{msc}{60E07}
\pmdefines{$n$-decomposable}
\pmdefines{$n$-divisible}
\pmdefines{infinitely divisible distribution}
\pmdefines{infinitely divisible}
\pmdefines{decomposable random variable}

\endmetadata

\usepackage{amssymb,amscd}
\usepackage{amsmath}
\usepackage{amsfonts}

% used for TeXing text within eps files
%\usepackage{psfrag}
% need this for including graphics (\includegraphics)
%\usepackage{graphicx}
% for neatly defining theorems and propositions
%\usepackage{amsthm}
% making logically defined graphics
%%\usepackage{xypic}
\usepackage{pst-plot}
\usepackage{psfrag}

% define commands here

\begin{document}
Let $n$ be a positive integer.  A real random variable $X$ defined on a probability space $(\Omega, \mathcal{F}, P)$ is said to be 
\begin{enumerate}
\item \emph{$n$-decomposable} if there exist $n$ independent random variables $X_1,\ldots,X_n$ such that $X$ is identically distributed as the sum $X_1+\cdots+X_n$.  A $2$-decomposable random variable is also called a \emph{decomposable random variable};
\item \emph{$n$-divisible} if $X$ is $n$-decomposable and the $X_i$'s can be chosen so that they are identically distributed;
\item \emph{infinitely divisible} if $X$ is $n$-divisible for every positive integer $n$.  In other words, $X$ can be written as the sum of $n$ iid random variables for any $n$.
\end{enumerate}

A distribution function is said to be \emph{infinitely divisible} if it is the distribution function of an infinitely divisible random variable.

\textbf{Remark}.  Any stable random variable is infinitely divisible.

Some examples of infinitely divisible distribution functions, besides those that are stable, are the gamma distributions, negative binomial distributions, and compound Poisson distributions.
%%%%%
%%%%%
\end{document}
