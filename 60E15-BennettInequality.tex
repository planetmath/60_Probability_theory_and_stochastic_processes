\documentclass[12pt]{article}
\usepackage{pmmeta}
\pmcanonicalname{BennettInequality}
\pmcreated{2013-03-22 16:12:25}
\pmmodified{2013-03-22 16:12:25}
\pmowner{Andrea Ambrosio}{7332}
\pmmodifier{Andrea Ambrosio}{7332}
\pmtitle{Bennett inequality}
\pmrecord{10}{38302}
\pmprivacy{1}
\pmauthor{Andrea Ambrosio}{7332}
\pmtype{Theorem}
\pmcomment{trigger rebuild}
\pmclassification{msc}{60E15}

% this is the default PlanetMath preamble.  as your knowledge
% of TeX increases, you will probably want to edit this, but
% it should be fine as is for beginners.

% almost certainly you want these
\usepackage{amssymb}
\usepackage{amsmath}
\usepackage{amsfonts}

% used for TeXing text within eps files
%\usepackage{psfrag}
% need this for including graphics (\includegraphics)
%\usepackage{graphicx}
% for neatly defining theorems and propositions
%\usepackage{amsthm}
% making logically defined graphics
%%%\usepackage{xypic}

% there are many more packages, add them here as you need them

% define commands here

\begin{document}
Theorem:(Bennett inequality, 1962):

Let $\{X_{i}\}_{i=1}^{n}$ be a collection of independent random
variables satisfying the conditions:

a) $E[X_{i}^{2}]<\infty $ $\forall i$, so that one can write $%
\sum_{i=1}^{n}E[X_{i}^{2}]=v^{2}$ \\
b) $\Pr\left\{\left\vert X_{i}\right\vert \leq M\right\} =1$ \ $\forall i$.

Then, for any $\varepsilon \geq 0$,
\[
\Pr\left\{ \sum_{i=1}^{n}\left( X_{i}-E[X_{i}]\right) >\varepsilon \right\}
\leq \exp \left[ -\frac{v^{2}}{M^{2}}\theta \left( \frac{\varepsilon M}{v^{2}%
}\right) \right] \leq \exp \left[ -\frac{\varepsilon }{2M}\ln \left( 1+\frac{%
\varepsilon M}{v^{2}}\right) \right] 
\]
where
\[
\theta \left( x\right) =\left( 1+x\right) \ln \left( 1+x\right) -x 
\]

Remark:
Observing that $\left( 1+x\right) \ln \left( 1+x\right) -x\geq 9\left( 1+%
\frac{x}{3}-\sqrt{1+\frac{2}{3}x}\right) \geq \frac{3x^{2}}{2\left(
x+3\right) }$ $\ \ \forall x\geq 0$, and plugging these expressions into the
bound, one obtains immediately the Bernstein inequality under the hypotheses of
boundness of random variables, as one might expect. However, Bernstein
inequalities, although weaker, hold under far more general hypotheses than
Bennett one.
%%%%%
%%%%%
\end{document}
