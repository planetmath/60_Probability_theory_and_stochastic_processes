\documentclass[12pt]{article}
\usepackage{pmmeta}
\pmcanonicalname{Mode}
\pmcreated{2013-03-22 14:23:33}
\pmmodified{2013-03-22 14:23:33}
\pmowner{CWoo}{3771}
\pmmodifier{CWoo}{3771}
\pmtitle{mode}
\pmrecord{4}{35889}
\pmprivacy{1}
\pmauthor{CWoo}{3771}
\pmtype{Definition}
\pmcomment{trigger rebuild}
\pmclassification{msc}{60A99}

% this is the default PlanetMath preamble.  as your knowledge
% of TeX increases, you will probably want to edit this, but
% it should be fine as is for beginners.

% almost certainly you want these
\usepackage{amssymb,amscd}
\usepackage{amsmath}
\usepackage{amsfonts}

% used for TeXing text within eps files
%\usepackage{psfrag}
% need this for including graphics (\includegraphics)
%\usepackage{graphicx}
% for neatly defining theorems and propositions
%\usepackage{amsthm}
% making logically defined graphics
%%%\usepackage{xypic}

% there are many more packages, add them here as you need them

% define commands here
\begin{document}
Given a probability distribution (density) function $f_X(x)$ with random variable $X$ and $x\in \mathbb{R}$, a \emph{mode} of $f_X(x)$ is a real number $\alpha$ such that:

\begin{enumerate}
\item
$f_X(\alpha)\neq \operatorname{min}(f_X(x))$,
\item
$f_X(\alpha)\geq f_X(z)$ for all $z\in \mathbb{R}$.
\end{enumerate}

\emph{The} mode of $f_X$ is the set of all modes of $f_X$ (It is also customary to say denote the mode of $f_X$ to be elements within the mode of $f_X$).  If the mode contains one element, then we say that $f_X$ is \emph{unimodal}.  If it has two elements, then $f_X$ is called \emph{bimodal}.  When $f_X$ has more than two modes, it is called \emph{multimodal}.

\begin{itemize}
\item
if $\Omega=\lbrace 0,1,2,2,3,4,4,4,5,5,6,7,8 \rbrace$ is the sample space for the random variable $X$, then the mode of the distribution function $f_X$ is 4.
\item
if $\Omega=\lbrace 0,2,4,5,6,6,7,9,11,11,14,18 \rbrace$ is the sample space for $X$, then the modes of $f_X$ are 6 and 11 and $f_X$ is bimodal.
\item
For a binomial distribution with mean $np$ and variance $np(1-p)$, the mode is $$\lbrace \alpha \mid p(n+1)-1\leq \alpha \leq p(n+1) \rbrace.$$
\item
For a Poisson distribution with integral sample space and mean $\lambda$, if $\lambda$ is non-integral, then the mode is the largest integer less than or equal to $\lambda$; if $\lambda$ is an integer, then both $\lambda$ and $\lambda-1$ are modes.
\item
For a normal distribution with mean $\mu$ and standard deviation $\sigma$, the mode is $\mu$.
\item
For a gamma distribution with the shape parameter $\gamma$, location parameter $\mu$, and scale parameter $\beta$, the mode is $\gamma-1$ if $\gamma>1$.
\item
Both the Pareto and the exponential distributions have mode = 0.
\end{itemize}
%%%%%
%%%%%
\end{document}
