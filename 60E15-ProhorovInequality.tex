\documentclass[12pt]{article}
\usepackage{pmmeta}
\pmcanonicalname{ProhorovInequality}
\pmcreated{2013-03-22 16:12:56}
\pmmodified{2013-03-22 16:12:56}
\pmowner{Andrea Ambrosio}{7332}
\pmmodifier{Andrea Ambrosio}{7332}
\pmtitle{Prohorov inequality}
\pmrecord{17}{38312}
\pmprivacy{1}
\pmauthor{Andrea Ambrosio}{7332}
\pmtype{Theorem}
\pmcomment{trigger rebuild}
\pmclassification{msc}{60E15}
\pmsynonym{Prokhorov inequality}{ProhorovInequality}

% this is the default PlanetMath preamble.  as your knowledge
% of TeX increases, you will probably want to edit this, but
% it should be fine as is for beginners.

% almost certainly you want these
\usepackage{amssymb}
\usepackage{amsmath}
\usepackage{amsfonts}

% used for TeXing text within eps files
%\usepackage{psfrag}
% need this for including graphics (\includegraphics)
%\usepackage{graphicx}
% for neatly defining theorems and propositions
%\usepackage{amsthm}
% making logically defined graphics
%%%\usepackage{xypic}

% there are many more packages, add them here as you need them

% define commands here
\DeclareMathOperator{\arsinh}{arsinh}

\begin{document}
Theorem (Prohorov inequality, 1959):

Let $\{X_{i}\}_{i=1}^{n}$ be a collection of independent random
variables satisfying the conditions:

a) $E[X_{i}^{2}]<\infty $ $\forall i$, so that one can write $%
\sum_{i=1}^{n}E[X_{i}^{2}]=v^{2}$ \\
b) $\Pr\left\{\left\vert X_{i}\right\vert \leq M\right\} =1$ \ $\forall i$.

Then, for any $\varepsilon \geq 0$,
\begin{eqnarray*}
\Pr\left\{ \sum_{i=1}^{n}\left( X_{i}-E[X_{i}]\right) >\varepsilon \right\}
&\leq &\exp \left[ -\frac{\varepsilon }{2M}\arsinh\left( \frac{\varepsilon
M}{2v^{2}}\right) \right] \\
\Pr\left\{ \left\vert \sum_{i=1}^{n}\left( X_{i}-E[X_{i}]\right) \right\vert
>\varepsilon \right\} &\leq &2\exp \left[ -\frac{\varepsilon }{2M}\arsinh\left( \frac{\varepsilon M}{2v^{2}}\right) \right]
\end{eqnarray*}

(See \PMlinkname{here}{AreaFunctions} for the meaning of $\arsinh(x)$)
%%%%%
%%%%%
\end{document}
