\documentclass[12pt]{article}
\usepackage{pmmeta}
\pmcanonicalname{IndependentIdenticallyDistributed}
\pmcreated{2013-03-22 14:27:29}
\pmmodified{2013-03-22 14:27:29}
\pmowner{CWoo}{3771}
\pmmodifier{CWoo}{3771}
\pmtitle{independent identically distributed}
\pmrecord{8}{35977}
\pmprivacy{1}
\pmauthor{CWoo}{3771}
\pmtype{Definition}
\pmcomment{trigger rebuild}
\pmclassification{msc}{60-00}
\pmsynonym{iid}{IndependentIdenticallyDistributed}
\pmsynonym{independent and identically distributed}{IndependentIdenticallyDistributed}
\pmdefines{identically distributed}

\usepackage{amssymb}
\usepackage{amsmath}
\usepackage{amsfonts}
\usepackage{amsthm}

% need this for including graphics (\includegraphics)
%\usepackage{graphicx}
% making logically defined graphics
%%%\usepackage{xypic}
\begin{document}
Two random variables $X$ and $Y$ are said to be \emph{identically distributed} if they are defined on the same probability space $(\Omega,\mathcal{F},P)$, and the distribution function $F_X$ of $X$ and the distribution function $F_Y$ of $Y$ are the same: $F_X=F_Y$.  When $X$ and $Y$ are identically distributed, we write $X \stackrel{d}{=} Y$.  

A set of random variables $X_i$, $i$ in some index set $I$, is identically distributed if $X_i \stackrel{d}{=} X_j$ for every pair $i,j\in I$.

A collection of random variables $X_i$ ($i\in I$) is said to be \emph{independent identically distributed}, if the $X_i$'s are identically distributed, and \PMlinkname{mutually independent}{Independent} (every finite subfamily of $X_i$ is independent). This is often abbreviated as \emph{iid}.

For example, the interarrival times $T_i$ of a Poisson process of rate $\lambda$ are independent and each have an exponential distribution with mean $1/\lambda$, so the $T_i$ are independent identically distributed random variables.

Many other examples are found in statistics, where individual data points are often assumed to realizations of iid random variables.
%%%%%
%%%%%
\end{document}
