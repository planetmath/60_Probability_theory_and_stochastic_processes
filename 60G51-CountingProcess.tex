\documentclass[12pt]{article}
\usepackage{pmmeta}
\pmcanonicalname{CountingProcess}
\pmcreated{2013-03-22 15:01:19}
\pmmodified{2013-03-22 15:01:19}
\pmowner{CWoo}{3771}
\pmmodifier{CWoo}{3771}
\pmtitle{counting process}
\pmrecord{5}{36730}
\pmprivacy{1}
\pmauthor{CWoo}{3771}
\pmtype{Definition}
\pmcomment{trigger rebuild}
\pmclassification{msc}{60G51}

\endmetadata

% this is the default PlanetMath preamble.  as your knowledge
% of TeX increases, you will probably want to edit this, but
% it should be fine as is for beginners.

% almost certainly you want these
\usepackage{amssymb,amscd}
\usepackage{amsmath}
\usepackage{amsfonts}

% used for TeXing text within eps files
%\usepackage{psfrag}
% need this for including graphics (\includegraphics)
%\usepackage{graphicx}
% for neatly defining theorems and propositions
%\usepackage{amsthm}
% making logically defined graphics
%%%\usepackage{xypic}

% there are many more packages, add them here as you need them

% define commands here
\begin{document}
A stochastic process $\lbrace X(t)\mid t\in
\mathbb{R}^{+}\cup\lbrace 0 \rbrace \rbrace$ is called a
\emph{counting process} if, for each outcome $\omega$ in the sample space $\Omega$,
\begin{enumerate}
\item $X(t)\in \mathbb{Z}^{+}\cup\lbrace 0 \rbrace$ for all $t$,
\item $X(t)(\omega)$ is piecewise constant,
\item $X(t)(\omega)$ is non-decreasing,
\item $X(t)(\omega)$ is right continuous (continuous from the right), and
\item for any $t$, there is an $s\in\mathbb{R}$ such that $t<s$ and $X(t)(\omega)+1=X(s)(\omega)$.
\end{enumerate}

\textbf{Remark}.  For any $t$, the random variable $X(t)$ is usually called the number of occurrences of some event by time $t$.  Then, for $s<t$,
$X(t)-X(s)$ is the number of occurrences in the half-open interval
$(s,t]$.
%%%%%
%%%%%
\end{document}
