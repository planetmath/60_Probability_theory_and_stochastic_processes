\documentclass[12pt]{article}
\usepackage{pmmeta}
\pmcanonicalname{LocalMartingale}
\pmcreated{2013-03-22 15:12:43}
\pmmodified{2013-03-22 15:12:43}
\pmowner{skubeedooo}{5401}
\pmmodifier{skubeedooo}{5401}
\pmtitle{local martingale}
\pmrecord{8}{36973}
\pmprivacy{1}
\pmauthor{skubeedooo}{5401}
\pmtype{Definition}
\pmcomment{trigger rebuild}
\pmclassification{msc}{60G07}
\pmclassification{msc}{60G48}
\pmrelated{Martingale}
\pmrelated{LocalPropertiesOfProcesses}

\endmetadata

% this is the default PlanetMath preamble.  as your knowledge
% of TeX increases, you will probably want to edit this, but
% it should be fine as is for beginners.

% almost certainly you want these
\usepackage{amssymb}
\usepackage{amsmath}
\usepackage{amsfonts}

% used for TeXing text within eps files
%\usepackage{psfrag}
% need this for including graphics (\includegraphics)
%\usepackage{graphicx}
% for neatly defining theorems and propositions
\usepackage{amsthm}
% making logically defined graphics
%%%\usepackage{xypic}

% there are many more packages, add them here as you need them

% define commands here
\begin{document}
\PMlinkescapeword{index set}
\PMlinkescapeword{constant}
\PMlinkescapeword{localization}
\PMlinkescapeword{theory}
\PMlinkescapeword{locally bounded}
\PMlinkescapeword{satisfy}

Let $(\Omega,\mathcal{F},(\mathcal{F}_t)_{t\in\mathbb{T}},\mathbb{P})$ be a filtered probability space, where the time index set $\mathbb{T}\subseteq\mathbb{R}$ has minimal element $t_0$. The most common cases are discrete-time, with $\mathbb{T}=\mathbb{Z}_+$, and continuous time where $\mathbb{T}=\mathbb{R}_+$, in which case $t_0=0$.

A process $X$ is said to be a \emph{local martingale} if it is \PMlinkname{locally}{LocalPropertiesOfProcesses} a right-continuous martingale. That is, if there is a sequence of stopping times $\tau_n$ almost surely increasing to infinity and such that the stopped processes $1_{\{\tau_n>t_0\}}X^{\tau_n}$ are martingales. Equivalently, $1_{\{\tau_n>t_0\}}X_{\tau_n\wedge t}$ is integrable and
\begin{equation*}
1_{\{\tau_n>t_0\}}X_{\tau_n\wedge s}=\mathbb{E}[1_{\{\tau_n>t_0\}}X_{\tau_n\wedge t}\mid\mathcal{F}_s]
\end{equation*}
for all $s<t\in\mathbb{T}$.
In the discrete-time case where $\mathbb{T}=\mathbb{Z}_+$ then it can be shown that a local martingale $X$ is a martingale if and only if $\mathbb{E}[|X_t|]<\infty$ for every $t\in\mathbb{Z}_+$.
More generally, in continuous-time where $\mathbb{T}$ is an interval of the real numbers, then the stronger property that
\begin{equation*}
\left\{X_{\tau}:\tau\le t\textrm{ is a stopping time}\right\}
\end{equation*}
is uniformly integrable for every $t\in\mathbb{T}$ gives a necessary and sufficient condition for a local martingale to be a martingale.

Local martingales form a very important class of processes in the theory of stochastic calculus. This is because the local martingale property is preserved by the stochastic integral, but the martingale property is not.
Examples of local martingales which are not proper martingales are given by solutions to the stochastic differential equation
\begin{equation*}
dX = X^{\alpha}\,dW
\end{equation*}
where $X$ is a nonnegative process, $W$ is a Brownian motion and $\alpha>1$ is a fixed real number.

An alternative definition of local martingales which is sometimes used requires $X^{\tau_n}$ to be a martingale for each $n$. This definition is slightly more restrictive, and is equivalent to the definition given above together with the condition that $X_{t_0}$ must be integrable.


%%%%%
%%%%%
\end{document}
