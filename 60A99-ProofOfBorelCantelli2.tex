\documentclass[12pt]{article}
\usepackage{pmmeta}
\pmcanonicalname{ProofOfBorelCantelli2}
\pmcreated{2013-03-22 14:29:35}
\pmmodified{2013-03-22 14:29:35}
\pmowner{kshum}{5987}
\pmmodifier{kshum}{5987}
\pmtitle{proof of Borel-Cantelli 2}
\pmrecord{4}{36028}
\pmprivacy{1}
\pmauthor{kshum}{5987}
\pmtype{Proof}
\pmcomment{trigger rebuild}
\pmclassification{msc}{60A99}

% this is the default PlanetMath preamble.  as your knowledge
% of TeX increases, you will probably want to edit this, but
% it should be fine as is for beginners.

% almost certainly you want these
\usepackage{amssymb}
\usepackage{amsmath}
\usepackage{amsfonts}

% used for TeXing text within eps files
%\usepackage{psfrag}
% need this for including graphics (\includegraphics)
%\usepackage{graphicx}
% for neatly defining theorems and propositions
%\usepackage{amsthm}
% making logically defined graphics
%%%\usepackage{xypic}

% there are many more packages, add them here as you need them

% define commands here
\begin{document}
Let $E$ denote the set of samples that are in $A_i$ infinitely often. We want to show that the complement of $E$ has probability zero.

As in the proof of Borel-Cantelli 1, we know that
\[
 E^c = \bigcup_{k=1}^\infty \bigcap_{i=k}^\infty A_i^c
\]
where the superscript $^c$ means set complement. But for each $k$,
\begin{align*}
P(\cap_{i=k} A_i^c) &= \prod_{i=k}^\infty P(A_i^c) \\
&= \prod_{i=k}^\infty (1-P(A_i))
\end{align*}
Here we use the assumption that the event $A_i$'s are independent. The inequality $1-a \leq e^{-a}$ and the assumption that the sum of $P(A_i)$ diverges together imply that
\[
  P(\cap_{i=k} A_i^c) \leq \exp(-\sum_{i=k}^\infty P(A_i)) = 0
\]
Therefore $E^c$ is a union of countable number of events, each of them has probability zero. So $P(E^c)=0$.
%%%%%
%%%%%
\end{document}
