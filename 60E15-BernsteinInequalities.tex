\documentclass[12pt]{article}
\usepackage{pmmeta}
\pmcanonicalname{BernsteinInequalities}
\pmcreated{2013-03-22 16:09:08}
\pmmodified{2013-03-22 16:09:08}
\pmowner{Andrea Ambrosio}{7332}
\pmmodifier{Andrea Ambrosio}{7332}
\pmtitle{Bernstein inequalities}
\pmrecord{21}{38233}
\pmprivacy{1}
\pmauthor{Andrea Ambrosio}{7332}
\pmtype{Theorem}
\pmcomment{trigger rebuild}
\pmclassification{msc}{60E15}

% this is the default PlanetMath preamble.  as your knowledge
% of TeX increases, you will probably want to edit this, but
% it should be fine as is for beginners.

% almost certainly you want these
\usepackage{amssymb}
\usepackage{amsmath}
\usepackage{amsfonts}

% used for TeXing text within eps files
%\usepackage{psfrag}
% need this for including graphics (\includegraphics)
%\usepackage{graphicx}
% for neatly defining theorems and propositions
%\usepackage{amsthm}
% making logically defined graphics
%%%\usepackage{xypic}

% there are many more packages, add them here as you need them

% define commands here

\begin{document}
1) Let $\{X_{i}\}_{i=1}^{n}$ be a collection of independent random
variables satisfying the conditions: \\
a) $E[X_{i}^{2}]<\infty$ $\forall i$, so that one can write $\sum_{i=1}^{n}E[X_{i}^{2}]=v^2$\\
b) $\exists c\in\mathbb{R}:\sum_{i=1}^{n}E[\left\vert X_{i}\right\vert ^{k}]\leq 
\frac{1}{2}k!v^2c^{k-2}$ for all integers $k\geq 3$

Then, for any $\varepsilon \geq 0$,
\[
\Pr\left\{ \sum_{i=1}^{n}\left( X_{i}-E[X_{i}]\right) >\varepsilon \right\}
\leq \exp \left[-\frac{v^2}{c^{2}}\left( 1+\frac{c\varepsilon }{v^2}-\sqrt{1+2%
\frac{c\varepsilon }{v^2}}\right) \right] \leq \exp \left( -\frac{\varepsilon
^{2}}{2\left( v^2+c\varepsilon \right) }\right) 
\]
\[
\Pr\left\{ \left\vert \sum_{i=1}^{n}\left( X_{i}-E[X_{i}]\right) \right\vert
>\varepsilon \right\} \leq 2\exp \left[-\frac{v^2}{c^{2}}\left( 1+\frac{%
c\varepsilon }{v^2}-\sqrt{1+2\frac{c\varepsilon }{v^2}}\right) \right] \leq
2\exp \left( -\frac{\varepsilon ^{2}}{2\left( v^2+c\varepsilon \right) }%
\right) 
\]

2) Let $\{X_{i}\}_{i=1}^{n}$ be a collection of independent, \PMlinkname{almost
surely absolutely bounded}{AlmostSurelyBoundedRandomVariable} random variables, that is $\Pr\left\{\left\vert X_{i}\right\vert \leq M\right\} =1\text{ \ }\forall i$. \\
Then, for any $\varepsilon \geq 0$,
\[
\Pr\left\{ \sum_{i=1}^{n}\left( X_{i}-E[X_{i}]\right) >\varepsilon \right\}
\leq \exp \left[-\frac{9v^2}{M^{2}}\left( 1+\frac{M\varepsilon }{3v^2}-\sqrt{1+2
\frac{M\varepsilon }{3v^2}}\right) \right] \leq \exp \left( -\frac{\varepsilon
^{2}}{2\left( v^2+\frac{M}{3}\varepsilon \right) }\right) 
\]
\[
\Pr\left\{ \left\vert \sum_{i=1}^{n}\left( X_{i}-E[X_{i}]\right) \right\vert
>\varepsilon \right\} \leq 2\exp \left[-\frac{9v^2}{M^{2}}\left( 1+\frac{
M\varepsilon }{3v^2}-\sqrt{1+2\frac{M\varepsilon }{3v^2}}\right) \right] \leq
2\exp \left( -\frac{\varepsilon ^{2}}{2\left( v^2+\frac{M}{3}\varepsilon
\right) }\right) 
\]
%%%%%
%%%%%
\end{document}
