\documentclass[12pt]{article}
\usepackage{pmmeta}
\pmcanonicalname{4Measurement}
\pmcreated{2014-04-22 19:36:22}
\pmmodified{2014-04-22 19:36:22}
\pmowner{rspuzio}{6075}
\pmmodifier{rspuzio}{6075}
\pmtitle{4. Measurement}
\pmrecord{6}{88088}
\pmprivacy{1}
\pmauthor{rspuzio}{6075}
\pmtype{Feature}

\endmetadata

% this is the default PlanetMath preamble.  as your knowledge
% of TeX increases, you will probably want to edit this, but
% it should be fine as is for beginners.

% almost certainly you want these
\usepackage{amssymb}
\usepackage{amsmath}
\usepackage{amsfonts}

% need this for including graphics (\includegraphics)
\usepackage{graphicx}
% for neatly defining theorems and propositions
\usepackage{amsthm}

% making logically defined graphics
%\usepackage{xypic}
% used for TeXing text within eps files
%\usepackage{psfrag}

% there are many more packages, add them here as you need them

% define commands here

\DeclareMathOperator{\Hom}{Hom}

\newcommand{\vecify}{{\mathcal V}}
\newcommand{\Act}{{A}}
\newcommand{\act}{{a}}
\newcommand{\Sit}{{S}}
\newcommand{\occ}{{v}}
\newcommand{\univ}{{\mathbf D}}
\newcommand{\uout}{{d_{out}}}
\newcommand{\uin}{{d_{in}}}
\newcommand{\mangle}{{\mathbf C}}

\newcommand{\psheaf}{{\mathcal F}}
\newcommand{\scat}{{\mathtt{Stoch}}}
\newcommand{\subs}{{\mathtt{Sys}}}
\newcommand{\mcat}{{\mathtt{Meas}}}
\newcommand{\eop}{{$\blacksquare$}}
\newcommand{\eod}{{${}$\\}}
\newcommand{\bra}{{\langle}}
\newcommand{\ket}{{\rangle}}

\newcommand{\cN}{{\mathcal N}}
\newcommand{\bR}{{\mathbb R}}
\newcommand{\fm}{{\mathfrak m}}
\newcommand{\cP}{{\mathcal P}}

\newtheorem{thm}{Theorem}
\newtheorem{prop}[thm]{Proposition}
\newtheorem{cor}[thm]{Corollary}

\theoremstyle{remark}
\newtheorem{eg}{Example}
\newtheorem{rem}{Remark}
\newtheorem{defn}{Definition}
\setcounter{eg}{1}
\setcounter{rem}{2}
\setcounter{defn}{7}
\setcounter{thm}{4}
\begin{document}
This section adapts %Definition~\ref{d:cmeasure}
\PMlinkname{Definition~1}{1introduction#Thmdefn1}
to distributed 
stochastic systems. The first step is to replace elements of 
state space $X$ with stochastic maps $\uin:\bR\rightarrow 
\vecify\Sit^\univ$, or equivalently probability distributions 
on $\Sit^\univ$, which are the system's inputs. Individual 
elements of $\Sit^\univ$ correspond to Dirac distributions. 

Second, replace function $f:X\rightarrow \bR$ with mechanism 
$\fm_\univ:\vecify\Sit^\univ\rightarrow \vecify\Act^\univ$. 
Since we are interested in the compositional structure of 
measurements we also consider submechanisms $\fm_\mangle$. 
However, comparing mechanisms requires that they have the same 
domain and range, so we extend $\fm_\mangle$ to the entire 
system as follows
\begin{equation}
    \label{e:extension}
	\fm_\mangle = \vecify \Sit^\univ\xrightarrow{\pi}
    \vecify\Sit^\mangle\xrightarrow{\fm_\mangle}
    \vecify \Act^\mangle\xrightarrow{\pi^\natural}
    \vecify \Act^\univ.
\end{equation}

We refer to the extension as $\fm_\mangle$ by abuse of notation.
We extend mechanisms implicitly whenever necessary without 
further comment. Extending mechanisms in this way maps the 
quale into a cloud of points in $\Hom(\vecify \Act^\univ,
\vecify\Sit^\univ)$ labeled by objects in $\subs_\univ$.

In the special case of the initial object $\bot_\univ$, define 
\begin{equation*}
		\label{e:null-extension}
	\fm_\bot = \vecify \Sit^\univ\xrightarrow{\omega}
    \bR\xrightarrow{\omega^\natural}\vecify \Act^\univ.
\end{equation*}
\begin{rem}
	\label{r:same-point}
	Subsystems differing by non-existent edges (%Remark~\ref{r:subsystems}
    \PMlinkname{Remark~2}{3distributeddynamicalsystems#Thmrem2}) 
    are mapped to the same mechanism by this construction, thus 
    making the fact that the edges do not exist explicit within 
    the formalism.
\end{rem}

Composing an input with a submechanism yields an output 
$\uout:=\fm_\mangle\circ \uin: \bR\rightarrow 
\vecify \Act^\univ$, which is a probability distribution on 
$\Act^\univ$. We are now in a position to define

\begin{defn}
	\label{d:stochmeas}
	A \emph{measuring device} is the dual $\fm^\natural_\mangle$
    to the mechanism of a subsystem. An \emph{output} is a 
    stochastic map $\uout:\bR\rightarrow \vecify\Act^\univ$. A 
    \emph{measurement} is a composition $\fm^\natural_\mangle
    \circ \uout:\bR\rightarrow \vecify\Sit^\univ$.
\end{defn}

%We switch freely between the three terms subsystem $\mangle$, mechanism $\fm_\mangle$ and measuring device $\fm_\mangle^\natural$ when no confusion is likely to arise.

Recall that stochastic maps of the form $\bR\rightarrow \vecify 
X$ correspond to probability distributions on $X$. Outputs as 
defined above are thus probability distributions on 
$\Act^\univ$, the output alphabet of $\univ$. Individual 
elements of $\Act^\univ$ are recovered as Dirac vectors: 
$\bR\xrightarrow{\delta_\act}\vecify\Act^\univ$.

\begin{defn}
	\label{d:ei}
	The \emph{effective information} generated by $\mangle_1$ 
    in the \emph{context} of subsystem $\mangle_2\subset 
    \mangle_1$ is
\begin{equation}
	\label{e:rel_ei}
	ei(\fm_{\mangle_2}\rightarrow \fm_{\mangle_1}, \uout) := 
	H\left[\fm_{\mangle_1}^\natural\circ \uout\Big\|
    \fm_{\mangle_2}^\natural\circ \uout\right].
\end{equation}
The \emph{null context}, corresponding to the empty subsystem 
$\bot=\emptyset\subset V_\univ\times V_\univ$, is a special 
case where $\fm_\mangle^\natural\circ \uout$ is replaced by the 
uniform distribution $\omega_\univ^\natural$ on $\Sit^\univ$. 
To simplify notation define
\begin{equation*}
	ei(\fm_\mangle,\uout):=ei(\fm_\bot\rightarrow\fm_\mangle,
    \uout).
\end{equation*}
\end{defn}

Here, $H[p\|q]=\sum_{i}p_i\log_2\frac{p_i}{q_i}$ is the 
Kullback-Leibler divergence or relative entropy 
\cite{jaynes:81}. Eq.~\eqref{e:rel_ei} expands as
\begin{equation}
    ei(\fm_{\mangle_2}\rightarrow \fm_{\mangle_1},\uout) 
	 = \sum_{s\in \Sit^\univ} \left\langle
     \fm^\natural_{\mangle_1}\circ \uout\Big|
     \delta_s\right\rangle
	\cdot \log_2 \frac{\left\langle\fm^\natural_{\mangle_1}
    \circ \uout\Big|\delta_s\right\rangle}
	{\left\langle\fm^\natural_{\mangle_2}\circ \uout\Big|
    \delta_s\right\rangle}.
\end{equation}
When $d_{out}=\delta_\act$ for some $\act\in \Act^\univ$ we 
have
\begin{equation}
	ei(\fm_{\mangle_2}\rightarrow \fm_{\mangle_1},\delta_\act)
	= \sum_{s\in\Sit^\univ} p_{\fm_{\mangle_1}}(s|\act)\cdot
    \log_2
	\frac{p_{\fm_{\mangle_1}}(s|\act)}{p_{\fm_{\mangle_2}}
    (s|\act)}.
\end{equation}

Definition~\ref{d:stochmeas} requires some unpacking. To relate
it to the classical notion of measurement, %Definition~\ref{d:cmeasure}
\PMlinkname{Definition~1}{1introduction#Thmdefn1}, 
we consider system $\univ=\left\{v_X\xrightarrow{f}v_Y\right\}$
where the alphabets of $v_X$ and $v_Y$ are the sets 
$\Act_{v_X}=X$ and $\Act_{v_Y}=Y$ respectively, and the 
mechanism of $v_Y$ is $\fm_Y=\vecify f$. In other words, 
system $\univ$ corresponds to a single deterministic function 
$f:X\rightarrow Y$.

\begin{prop}
	[classical measurement]
	\label{t:classmeas}
	The measurement $(\vecify f)^\natural\circ \delta_y$ 
    performed when deterministic function $f:X\rightarrow Y$ 
    outputs $y$ is equivalent to the preimage $f^{-1}(y)$. 
    Effective information is $ei(\vecify f,\delta_y)=
    \log_2\frac{|X|}{|f^{-1}(y)|}$.
\end{prop}

\noindent
Proof:
By %Corollary~\ref{t:preimage} 
\PMlinkname{Corollary~2}{2stochasticmaps#Thmthm2}
measurement $(\vecify f)^\natural\circ \delta_y$ is conditional
distribution
\begin{equation*}
	p_{\vecify f}(x|y) = \left\{\begin{matrix}
		\frac{1}{|f^{-1}(y)|} & \mbox{if } f(x)=y\\
		0 &\mbox{else}.
	\end{matrix}\right.
\end{equation*}
which generalizes the preimage. Effective information follows 
immediately.
\eop

Effective information can be interpreted as quantifying a 
measurement's precision. It is high if few inputs cause $f$ 
to output $y$ out of many  -- i.e. $f^{-1}(y)$ has few 
elements relative to $|X|$  -- and conversely is low if many 
inputs cause $f$ to output $y$ -- i.e. if the output is 
relatively insensitive to changes in the input. Precise 
measurements say a lot about what the input could have been 
and conversely for vague measurements with low $ei$.

The point of this paper is to develop techniques for studying 
measurements constructed out of two or more functions. We 
therefore present computations for the simplest case, 
distributed system $X\times Y\xrightarrow{g}Z$, in considerable
detail. Let $\univ$ be the graph
\begin{figure}
 \centering
 \includegraphics{information-theoretic-distributed-measurement-4.1.png}
\end{figure}
%\begin{equation*}
%    \xymatrix{
%	v_X\ar[dr] & & v_Y\ar[dl]\\
%	& v_Z
%	}
%\end{equation*}
with obvious assignments of alphabets and the mechanism of 
$v_Z$ as $\fm_Z=\vecify g$. To make the formulas more readable 
let $\fm_{XY}=\vecify g$, $\fm_{X\bullet}=
\vecify g\circ\pi^\natural_{XY,X}$ and  $\fm_{\bullet Y}=
\vecify g\circ\pi^\natural_{XY,Y}$. We then obtain lattice 
\begin{figure}
 \centering
 \includegraphics{information-theoretic-distributed-measurement-4.2.png}
\end{figure}
%\begin{equation}
%	\label{e:diag}
%	\xymatrix{
%		& \top=\fm_{XY} \\
%		\fm_{X\bullet}\ar[ur]^{ei(\fm_{X\bullet}\rightarrow\fm_{XY})} & & \fm_{\bullet Y}\ar[ul]_{ei(\fm_{\bullet Y}\rightarrow \fm_{XY})} \\
%		& \fm_\bot\ar[ul]^{ei(\fm_{X\bullet})}\ar[ur]_{ei(\fm_{\bullet Y})}\ar[uu]^{ei(\fm_{XY})}
%	}
%\end{equation}
The remainder of this section and most of the next analyzes 
measurements in the lattice.

\begin{prop}
	[partial measurement]
	\label{t:confmeas}
	The measurement performed on $X$ when $g:X\times Y
    \rightarrow Z$  outputs $z$, treating $Y$ as extrinsic 
    noise, is conditional distribution
	\begin{equation}
		\label{e:conf-preimage}
		p(x|z) = \left\{\begin{matrix}
			\frac{|g_{x\times Y}^{-1}(z)|}{|g^{-1}(z)|} &
			\mbox{if } g(x,y)=z\mbox{ for some }y\in Y\\
			0 & \mbox{else,}
		\end{matrix}\right.
	\end{equation}
	where $g^{-1}_{x\times Y}(z):= pr_Y(g^{-1}(z)\cap \{x\}
    \times Y)$. The effective information generated by the
    partial measurement is
	\begin{equation}
		\label{e:conf-ei}
		ei\big(\fm_{X\bullet}^\natural,\delta_z\big) 		
		= \log_2|X|+\sum_{x\in X}p(x|z)\cdot
		\log_2 p(x|z).
%		ei\big((\vecify g\circ \pi_{XY,X}^\natural)^\natural,
\delta_z\big) 		
%		= \log_2|X|+\sum_{x\in X}\frac{|g^{-1}_{x\times Y}(z)|}
{|g^{-1}(z)|}\cdot
%		\log_2\frac{|g^{-1}_{x\times Y}(z)|}{|g^{-1}(z)|}.
	\end{equation}
\end{prop}

\noindent
Proof: Treating $Y$ as a source of extrinsic noise yields 
$\vecify X\xrightarrow{\pi^\natural}\vecify X\otimes 
\vecify Y\xrightarrow{\vecify g}\vecify Z$ which takes 
$\delta_x\mapsto \frac{1}{|Y|}\sum_{y\in Y}\delta_{g(x,y)}$. 
The dual is
\begin{equation*}
    \fm_{X\bullet}^\natural=\pi_{XY,X}\circ 
    (\vecify g)^\natural:\delta_z\mapsto \sum_{x\in X}
    \frac{|g^{-1}_{x\times Y}(z)|}{|g^{-1}(z)|}\cdot\delta_x.
\end{equation*}	
The computation of effective information follows immediately. 
\eop

A partial measurement is precise if the preimage $g^{-1}(z)$ 
has small or empty intersection with $\{x\}\times Y$ for most 
$x$, and large intersection for few $x$.

Propositions~\ref{t:classmeas} and \ref{t:confmeas} compute 
effective information of a measurement relative to the null 
context provided by complete ignorance (the uniform 
distribution). We can also compute the effective information 
generated by a measurement in the context of a submeasurement:

\begin{prop}
	[relative measurement]
	\label{t:relmeas}
	The information generated by measurement $X\times Y\xrightarrow{g}Z$ in the context of the partial measurement where $Y$ is unobserved noise, is
	\begin{equation}
		\label{e:relmeas}
		ei(\fm_{X\bullet}\rightarrow \fm_{XY},\delta_z)=
		\sum_{x\in X} \frac{g^{-1}_{x\times Y}(z)}{g^{-1}(z)}\log_2
		\frac{|Y|}{g^{-1}_{x\times Y}(z)}.
	\end{equation}
\end{prop}

\noindent
Proof: Applying Propositions~\ref{t:classmeas} and 
\ref{t:confmeas} obtains
\begin{equation*}
	ei(\fm_{X\bullet}\rightarrow \fm_{XY},\delta_z)=
	\sum_{(x,y)\in g^{-1}(z)}\frac{1}{|g^{-1}(z)|}
    \log_2\left[
	\frac{1}{|g^{-1}(z)|}\cdot\frac{|g^{-1}(z)|\cdot|Y|}
    {|g^{-1}_{x\times Y}(z)|}
	\right]
\end{equation*}
which simplifies to the desired expression. 
\eop

To interpret the result decompose $X\times Y\xrightarrow{g} Z$ 
into a family of functions $\mathcal{R}=
\left\{Y\xrightarrow{g_{x\times Y}}Z\big|x\in X\right\}$ 
labeled by elements of $X$, where $g_{x\times Y}(y):=g(x,y)$. 
The precision of the measurement performed by $g_{x\times Y}$ 
s $\log_2\frac{|Y|}{g^{-1}_{x\times Y}(z)}$. It follows that 
the precision of the relative measurement, 
Eq.~\eqref{e:relmeas}, is the expected precision of the 
measurements performed by family $\mathcal{R}$ taken with 
respect to the probability distribution 
$p(x|z)=\frac{g^{-1}_{x\times Y}(z)}{g^{-1}(z)}$ generated by
the noisy measurement.

In the special case of $g:X\times Y\rightarrow Z$ relative 
precision is simply the difference of the precision of the 
larger and smaller subsystems:

\begin{cor}[comparing measurements]
	\label{t:diffmeas}
	\begin{equation*}
		ei(\fm_{X\bullet}\rightarrow \fm_{XY},\delta_z) = 
        ei(\fm_{XY},\delta_z) -ei(\fm_{X\bullet},\delta_z)
	\end{equation*}
\end{cor}

\noindent
Proof: Applying Propositions~\ref{t:classmeas}, 
\ref{t:confmeas}, \ref{t:relmeas} and simplifying obtains
\begin{align*}
    ei(\fm_{XY},\delta_z) -ei(\fm_{X\bullet},\delta_z) 
	& = \log_2\frac{|X|\cdot|Y|}{|g^{-1}(z)|}
	-\sum_x \frac{|g^{-1}_{x\times Y}(z)|}{|g^{-1}(z)|}\log_2\frac{|X|\cdot |g^{-1}_{x\times Y}(z)|}{|g^{-1}(z)|}\\
	& = \log_2 \frac{|Y|}{|g^{-1}(z)|}+\sum_{(x,y)\in g^{-1}(z)}\frac{1}{|g^{-1}(z)|}
	\log_2 \frac{|g^{-1}(z)|}{|g^{-1}_{x\times Y}(z)|} \\
	& = ei(\fm_{X\bullet}\rightarrow \fm_{XY},\delta_z).
	\,\,\blacksquare
\end{align*}

\begin{thebibliography}{10}
\providecommand{\bibitemdeclare}[2]{}
\providecommand{\urlprefix}{Available at }
\providecommand{\url}[1]{\texttt{#1}}
\providecommand{\href}[2]{\texttt{#2}}
\providecommand{\urlalt}[2]{\href{#1}{#2}}
\providecommand{\doi}[1]{doi:\urlalt{http://dx.doi.org/#1}{#1}}
\providecommand{\bibinfo}[2]{#2}

%\bibitemdeclare{inproceedings}{jaynes:81}
\bibitem{jaynes:81}
\bibinfo{author}{E~T Jaynes} (\bibinfo{year}{1985}):
  \emph{\bibinfo{title}{Entropy and {S}earch {T}heory}}.
\newblock In \bibinfo{editor}{CR~Smith} \& \bibinfo{editor}{WT~Grandy},
  editors: {\sl \bibinfo{booktitle}{Maximum-entropy and {B}ayesian {M}ethods in
  {I}nverse {P}roblems}}, \bibinfo{publisher}{Springer}.

\end{thebibliography}
\end{document}
