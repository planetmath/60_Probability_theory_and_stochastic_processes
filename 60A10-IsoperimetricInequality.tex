\documentclass[12pt]{article}
\usepackage{pmmeta}
\pmcanonicalname{IsoperimetricInequality}
\pmcreated{2013-03-22 14:02:47}
\pmmodified{2013-03-22 14:02:47}
\pmowner{bbukh}{348}
\pmmodifier{bbukh}{348}
\pmtitle{isoperimetric inequality}
\pmrecord{12}{35397}
\pmprivacy{1}
\pmauthor{bbukh}{348}
\pmtype{Theorem}
\pmcomment{trigger rebuild}
\pmclassification{msc}{60A10}
\pmclassification{msc}{51M25}
\pmclassification{msc}{51M16}

\endmetadata

\usepackage{amssymb}
\usepackage{amsmath}
\usepackage{amsfonts}

% used for TeXing text within eps files
%\usepackage{psfrag}
% need this for including graphics (\includegraphics)
%\usepackage{graphicx}
% for neatly defining theorems and propositions
%\usepackage{amsthm}
% making logically defined graphics
%%%\usepackage{xypic}

\DeclareMathOperator{\vol}{vol}
\makeatletter
\@ifundefined{bibname}{}{\renewcommand{\bibname}{References}}
\makeatother
\begin{document}
\PMlinkescapeword{states}
\PMlinkescapeword{terms}
The classical isoperimetric inequality says that if a planar
figure has perimeter $P$ and area $A$, then
\begin{equation*}
4\pi A\leq P^2,
\end{equation*}
where the equality holds if and only if the figure is a circle.
That is, the circle is the figure that encloses the largest area
among all figures of same perimeter.

The analogous statement is true in arbitrary dimension. The
$d$-dimensional ball has the largest volume among all figures of
equal surface area.

The isoperimetric inequality can alternatively be stated using the
$\epsilon$-neighborhoods. An $\epsilon$-neighborhood of a set $S$,
denoted here by $S_\epsilon$, is the set of all points whose
distance to $S$ is at most $\epsilon$. The isoperimetric
inequality in terms of $\epsilon$-neighborhoods states that
$\vol(S_\epsilon)\geq \vol(B_\epsilon)$ where $B$ is the ball of
the same volume as $S$. The classical isoperimetric inequality can
be recovered by taking the limit $\epsilon\to 0$. 
The advantage of this formulation is that it
does not depend on the notion of surface area, and so can be
generalized to arbitrary measure spaces with a metric.

An example when this general formulation proves useful is the
Talagrand's isoperimetric theory dealing with \PMlinkname{Hamming}{HammingDistance}-like
distances in product spaces. The theory has proven to be very
useful in many applications of probability to combinatorics.

\begin{thebibliography}{1}

\bibitem{cite:alon_spencer_probmethod}
Noga Alon and Joel~H. Spencer.
\newblock {\em The probabilistic method}.
\newblock John Wiley \& Sons, Inc., second edition, 2000.
\newblock \PMlinkexternal{Zbl 0996.05001}{http://www.emis.de/cgi-bin/zmen/ZMATH/en/quick.html?type=html&an=0996.05001}.

\bibitem{cite:matousek_discgeom}
Ji{\v{r}}{\'\i} Matou{\v{s}}ek.
\newblock {\em Lectures on Discrete Geometry}, volume 212 of {\em GTM}.
\newblock Springer, 2002.
\newblock \PMlinkexternal{Zbl 0999.52006}{http://www.emis.de/cgi-bin/zmen/ZMATH/en/quick.html?type=html&an=0999.52006}.

\end{thebibliography}
%%%%%
%%%%%
\end{document}
