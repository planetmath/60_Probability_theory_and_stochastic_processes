\documentclass[12pt]{article}
\usepackage{pmmeta}
\pmcanonicalname{OptionalProcess}
\pmcreated{2013-03-22 18:37:34}
\pmmodified{2013-03-22 18:37:34}
\pmowner{gel}{22282}
\pmmodifier{gel}{22282}
\pmtitle{optional process}
\pmrecord{5}{41363}
\pmprivacy{1}
\pmauthor{gel}{22282}
\pmtype{Definition}
\pmcomment{trigger rebuild}
\pmclassification{msc}{60G07}
%\pmkeywords{stochastic process}
\pmrelated{ProgressivelyMeasurableProcess}
\pmrelated{PredictableProcess}
\pmdefines{optional}

\endmetadata

% almost certainly you want these
\usepackage{amssymb}
\usepackage{amsmath}
\usepackage{amsfonts}

% used for TeXing text within eps files
%\usepackage{psfrag}
% need this for including graphics (\includegraphics)
%\usepackage{graphicx}
% for neatly defining theorems and propositions
\usepackage{amsthm}
% making logically defined graphics
%%%\usepackage{xypic}

% there are many more packages, add them here as you need them

% define commands here
\newtheorem*{theorem*}{Theorem}
\newtheorem*{lemma*}{Lemma}
\newtheorem*{corollary*}{Corollary}
\newtheorem*{definition*}{Definition}
\newtheorem{theorem}{Theorem}
\newtheorem{lemma}{Lemma}
\newtheorem{corollary}{Corollary}
\newtheorem{definition}{Definition}

\begin{document}
\PMlinkescapeword{property}
\PMlinkescapeword{index set}
\PMlinkescapeword{equivalence}
\PMlinkescapeword{interval}
\PMlinkescapeword{filtration}

Suppose we are given a \PMlinkname{filtration}{FiltrationOfSigmaAlgebras} $(\mathcal{F})_{t\in\mathbb{T}}$ on a measurable space $(\Omega,\mathcal{F})$. A stochastic process is said to be adapted if $X_t$ is $\mathcal{F}_t$-measurable for every time $t$ in the index set $\mathbb{T}$. For an arbitrary, uncountable, index set $\mathbb{T}\subseteq\mathbb{R}$, this property is too restrictive to be useful. Instead, we can impose measurability conditions on $X$ considered as a map from $\mathbb{T}\times\Omega$ to $\mathbb{R}$.
For instance, we could require $X$ to be progressively measurable, but that is still too weak a condition for many purposes. A stronger condition is for $X$ to be \emph{optional}. The index set $\mathbb{T}$ is assumed to be a closed subset of $\mathbb{R}$ in the following definition.

The class of optional processes forms the smallest set containing all adapted and right-continuous processes, and which is closed under taking limits of sequences of processes.

The $\sigma$-algebra, $\mathcal{O}$, on $\mathbb{T}\times\Omega$ generated by the right-continuous and adapted processes is called the \emph{optional} $\sigma$-algebra. Then, a process is optional if and only if it is $\mathcal{O}$-measurable.

Alternatively, the optional $\sigma$-algebra may be defined as
\begin{equation*}
\mathcal{O}=\sigma\left(\left\{[T,\infty):T\textrm{ is a stopping time}\right\}\right).
\end{equation*}
Here, $[T,\infty)$ is a stochastic interval, consisting of the pairs $(t,\omega)\in\mathbb{T}\times\Omega$ such that $T(\omega)\le t$.
In continuous-time, the equivalence of these two definitions for $\mathcal{O}$ does require mild conditions on the filtration --- it is enough for $\mathcal{F}_t$ to be universally complete.

In the discrete-time case where the index set $\mathbb{T}$ countable, then the definitions above imply that a process $X_t$ is optional if and only if it is adapted.

%%%%%
%%%%%
\end{document}
