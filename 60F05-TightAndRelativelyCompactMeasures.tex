\documentclass[12pt]{article}
\usepackage{pmmeta}
\pmcanonicalname{TightAndRelativelyCompactMeasures}
\pmcreated{2013-03-22 17:19:37}
\pmmodified{2013-03-22 17:19:37}
\pmowner{fernsanz}{8869}
\pmmodifier{fernsanz}{8869}
\pmtitle{tight and relatively compact measures}
\pmrecord{6}{39679}
\pmprivacy{1}
\pmauthor{fernsanz}{8869}
\pmtype{Definition}
\pmcomment{trigger rebuild}
\pmclassification{msc}{60F05}
%\pmkeywords{tight measures}
%\pmkeywords{weak convergence measures}
%\pmkeywords{Helly's theoerem}
%\pmkeywords{Levy's theorem}
%\pmkeywords{Prokhorov  theorem}
\pmrelated{LindebergsCentralLimitTheorem}

\endmetadata

% this is the default PlanetMath preamble.  as your knowledge
% of TeX increases, you will probably want to edit this, but
% it should be fine as is for beginners.

% almost certainly you want these
\usepackage{amssymb}
\usepackage{amsmath}
\usepackage{amsfonts}
\usepackage{amsthm}

% define commands here
\newtheorem*{thm}{Theorem}
\theoremstyle{definition}
\newtheorem{defn}{Definition}
\newcommand{\B}{\mathcal{B}}
\newcommand{\M}{\mathcal{M}}
\begin{document}
\title{Tight and relatively compact measures}%
\author{Fernando Sanz}%

\begin{defn}
Let $\M=\{\mu_i,i \in I \}$ be a family of finite measures on the
Borel subsets of a metric space $\Omega$. We say that $\M$ is tight
iff for each $\epsilon>0$ there is a compact set $K$ such that
$\mu_i(\Omega-K)<\epsilon$ for all $i$. We say that $\M$ is
relatively compact iff each sequence in $\M$ has a subsequence
converging weakly to a finite measure on $\B(\Omega)$.

If $\{F_i,i \in I \}$ is a family of distribution functions,
relative compactness or tightness of $\{F_i\}$ refers to relative
compactness or tightness of the corresponding measures.
\end{defn}

\medskip

\begin{thm}
Let $\{F_i,i \in I \}$ be a family of distribution functions with
$F_i(\infty)-F_i(-\infty)<M<\infty$ for all  $i$. The family is
tight iff it is relatively compact.
\end{thm}

\begin{proof}
Coming soon...(needs other theorems before)
\end{proof}
%%%%%
%%%%%
\end{document}
