\documentclass[12pt]{article}
\usepackage{pmmeta}
\pmcanonicalname{UcpConvergenceOfProcesses}
\pmcreated{2013-03-22 18:39:48}
\pmmodified{2013-03-22 18:39:48}
\pmowner{gel}{22282}
\pmmodifier{gel}{22282}
\pmtitle{ucp convergence of processes}
\pmrecord{6}{41406}
\pmprivacy{1}
\pmauthor{gel}{22282}
\pmtype{Definition}
\pmcomment{trigger rebuild}
\pmclassification{msc}{60G05}
\pmclassification{msc}{60G07}
%\pmkeywords{stochastic process}
%\pmkeywords{convergence in probability}
\pmrelated{CadlagProcess}
\pmrelated{SemimartingaleTopology}
\pmdefines{ucp convergence}
\pmdefines{ucp topology}
\pmdefines{converges ucp}

% almost certainly you want these
\usepackage{amssymb}
\usepackage{amsmath}
\usepackage{amsfonts}

% used for TeXing text within eps files
%\usepackage{psfrag}
% need this for including graphics (\includegraphics)
%\usepackage{graphicx}
% for neatly defining theorems and propositions
\usepackage{amsthm}
% making logically defined graphics
%%%\usepackage{xypic}

% there are many more packages, add them here as you need them

% define commands here
\newtheorem*{theorem*}{Theorem}
\newtheorem*{lemma*}{Lemma}
\newtheorem*{corollary*}{Corollary}
\newtheorem*{definition*}{Definition}
\newtheorem{theorem}{Theorem}
\newtheorem{lemma}{Lemma}
\newtheorem{corollary}{Corollary}
\newtheorem{definition}{Definition}

\begin{document}
\PMlinkescapeword{mode}
\PMlinkescapeword{theory}
\PMlinkescapeword{frequently in}
\PMlinkescapeword{restricted}
\PMlinkescapeword{generated by}
\PMlinkescapeword{intervals}
\PMlinkescapeword{solution}
\PMlinkescapeword{discrete}
\PMlinkescapeword{partition}
\PMlinkescapeword{property}
\PMlinkescapeword{equality}
\PMlinkescapeword{probability space}
\PMlinkescapeword{well defined}
\PMlinkescapeword{implies}
\PMlinkescapeword{limits}
\PMlinkescapeword{cadlag}
\PMlinkescapeword{limits}

Let $(\Omega,\mathcal{F},(\mathcal{F}_t)_{t\in\mathbb{R}_+},\mathbb{P})$ be a filtered probability space. Then a sequence of stochastic processes $(X^n_t)_{t\in\mathbb{R}_+}$ is said to converge to the process $(X_t)$ in the \emph{ucp topology} (uniform convergence on compacts in probability) if
\begin{equation}\label{eq:1}
\sup_{s< t}|X^n_s-X_s|\rightarrow 0
\end{equation}
\PMlinkname{in probability}{ConvergenceInProbability} as $n\rightarrow\infty$, for every $t> 0$. That is, if
\begin{equation*}
\mathbb{P}(\sup_{s< t}|X^n_s-X_s|>\epsilon)\rightarrow 0
\end{equation*}
for all $\epsilon,t>0$.
The notation $X^n\xrightarrow{\rm ucp} X$ is sometimes used, and $X^n$ is said to converge ucp to $X$.
This mode of convergence occurs frequently in the theory of continuous-time stochastic processes, and some examples are given below.

Note that the expression on the left hand side of (\ref{eq:1}) is a supremum of an uncountable set of random variables and, therefore, need not be a measurable quantity in general.
If, however, the processes have left or right-continuous sample paths, then the supremum can be restricted to rational times
\begin{equation*}
\sup_{s< t}|X^n_s-X_s|
=\sup_{\substack{s<t,\\s\in\mathbb{Q}_+}}|X^n_s-X_s|
\end{equation*}
and is measurable. Typically, it is only required that the sample paths are left or right-continuous almost surely, so the above equality holds on a set of probability one.
More generally, if the processes are jointly measurable, then the expression on the left hand side of (\ref{eq:1}) will be a measurable random variable in the \PMlinkname{completion}{CompleteMeasure} of the probability space (by the measurable projection theorem).
This is enough to ensure that the definition above is meaningful, and gives a well defined topology on the space of jointly measurable processes.

The ucp topology can be generated by a pseudometric. For example, setting
\begin{equation*}
D^{\rm ucp}(X)=\sum_{n=1}^\infty 2^{-n}\mathbb{E}[\min(1,\sup_{s<n}|X_s|)]
\end{equation*}
then $(X,Y)\mapsto D^{\rm ucp}(X-Y)$ is a pseudometric on the space of measurable processes such that $X^n\xrightarrow{\rm ucp} X$ if and only if $D^{\rm ucp}(X^n-X)\rightarrow 0$ as $n\rightarrow\infty$.
Furthermore, this becomes a metric under the identification of processes with almost surely identical sample paths.


If $X^n\xrightarrow{\rm ucp} X$ then we may pass to a subsequence satisfying $D(X^{n_{k}}-X)<2^{-k}$. The Borel-Cantelli lemma then implies that $X^{n_k}\rightarrow X$ uniformly on all compact intervals, with probability one.
So, any sequence converging in the ucp topology has a subsequence converging uniformly on compacts, with probability one. Consequently, given any property of the sample paths which is preserved under uniform convergence on compacts, then it is also preserved under ucp convergence with probability one. In particular, ucp limits of cadlag processes are themselves cadlag.

Some examples of ucp convergence are given below.

\begin{enumerate}
\item
If $X^n,X$ are cadlag martingales such that $\mathbb{E}[|X^n_t-X_t|]\rightarrow 0$ for every $t>0$ then, Doob's inequality
\begin{equation*}
\mathbb{P}\left(\sup_{s\le t}|X^n_s-X_s|>\epsilon\right)\le\epsilon^{-1}\mathbb{E}[|X^n_t-X_t|]
\end{equation*}
shows that $X^n$ converges ucp to $X$.

\item Let $X$ be a semimartingale and $\xi^n$ be predictable processes converging pointwise to $\xi$ such that $\sup_n|\xi^n|$ is $X$-integrable. Then, the \PMlinkname{dominated convergence theorem for stochastic integration}{DominatedConvergenceForStochasticIntegration} gives
\begin{equation*}
\int\xi^n\,dX\xrightarrow{\rm ucp}\int\xi\,dX.
\end{equation*}

\item
Suppose that the stochastic differential equation
\begin{equation*}
dX=a(X)\,dW+b(X)\,dt
\end{equation*}
for continuous functions $a,b\colon\mathbb{R}\rightarrow\mathbb{R}$ and Brownian motion $W$ has a unique solution with $X_0=0$. For any partition $0=t_0\le t_1\le\cdots\uparrow\infty$, the following discrete approximation can be constructed
\begin{equation*}
\tilde X_0=0,\ \tilde X_{t_{k+1}}=\tilde X_{t_k}+a(\tilde X_{t_k})(W_{t_{k+1}}-W_{t_k})+b(\tilde X_{t_k})(t_{k+1}-t_{k}).
\end{equation*}
Setting $\tilde X_t=\tilde X_{t_k}$ for $t\in(t_k,t_{k+1})$ then these discrete approximations converge to $X$ in the ucp topology as the partition mesh goes to zero.
\end{enumerate}


\begin{thebibliography}{9}
\bibitem{protter}
Philip E. Protter, \emph{Stochastic integration and differential equations.} Second edition. Applications of Mathematics, 21. Stochastic Modelling and Applied Probability. Springer-Verlag, 2004.
\end{thebibliography}

%%%%%
%%%%%
\end{document}
