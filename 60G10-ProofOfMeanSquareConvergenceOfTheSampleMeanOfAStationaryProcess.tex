\documentclass[12pt]{article}
\usepackage{pmmeta}
\pmcanonicalname{ProofOfMeanSquareConvergenceOfTheSampleMeanOfAStationaryProcess}
\pmcreated{2013-03-22 15:22:19}
\pmmodified{2013-03-22 15:22:19}
\pmowner{georgiosl}{7242}
\pmmodifier{georgiosl}{7242}
\pmtitle{proof of mean square convergence of the sample mean of a stationary process}
\pmrecord{6}{37198}
\pmprivacy{1}
\pmauthor{georgiosl}{7242}
\pmtype{Proof}
\pmcomment{trigger rebuild}
\pmclassification{msc}{60G10}

\endmetadata

% this is the default PlanetMath preamble.  as your knowledge
% of TeX increases, you will probably want to edit this, but
% it should be fine as is for beginners.

% almost certainly you want these
\usepackage{amssymb}
\usepackage{amsmath}
\usepackage{amsfonts}

% used for TeXing text within eps files
%\usepackage{psfrag}
% need this for including graphics (\includegraphics)
%\usepackage{graphicx}
% for neatly defining theorems and propositions
%\usepackage{amsthm}
% making logically defined graphics
%%%\usepackage{xypic}

% there are many more packages, add them here as you need them

% define commands here
\def\var{\operatorname{var}}
\def\cov{\operatorname{cov}}
\begin{document}
$$n\var(\bar X_n)=\frac{1}{n}\sum_{i=1}^{n}\sum_{j=1}^{n}\cov(X_i,X_j)\\=
\sum_{|h|<n}(1-\frac{|h|}{n})\gamma(h)\leq \sum_{|h|<n}|\gamma(h)|$$
\\If $\gamma(n) \to 0$ as  $n \to \infty$ then $\lim_{n \to \infty }\frac{1}{n}\sum_{|h|<n}|\gamma(h)|=2\lim_{n \to \infty }|\gamma(n)|=0$, whence 
$\var[\bar X_n] \to 0$.
\\If $\sum_{h=-\infty}^{\infty}|\gamma(h)|<\infty$ then the dominated Convergence theorem gives 
$$\lim_{n \to \infty }\sum_{|h|<n}(1-\frac{|h|}{n})\gamma(h)=\sum_{h=-\infty}^{\infty}\gamma(h)$$.
%%%%%
%%%%%
\end{document}
