\documentclass[12pt]{article}
\usepackage{pmmeta}
\pmcanonicalname{ItosFormula}
\pmcreated{2013-03-22 17:16:14}
\pmmodified{2013-03-22 17:16:14}
\pmowner{stevecheng}{10074}
\pmmodifier{stevecheng}{10074}
\pmtitle{It\^o's formula}
\pmrecord{13}{39610}
\pmprivacy{1}
\pmauthor{stevecheng}{10074}
\pmtype{Axiom}
\pmcomment{trigger rebuild}
\pmclassification{msc}{60H10}
\pmclassification{msc}{60H05}
\pmsynonym{It\^o's formula}{ItosFormula}
\pmsynonym{It\^o's chain rule}{ItosFormula}
\pmsynonym{Ito's formula}{ItosFormula}
\pmsynonym{Ito's lemma}{ItosFormula}
\pmsynonym{Ito's chain rule}{ItosFormula}
\pmrelated{ItosLemma2}
\pmrelated{GeneralizedItoFormula}

\endmetadata

% The standard font packages
\usepackage{amssymb}
\usepackage{amsmath}
\usepackage{amsfonts}

% For neatly defining theorems and definitions
%\usepackage{amsthm}

% Including EPS/PDF graphics (\includegraphics)
%\usepackage{graphicx}

% Making matrix-based graphics
%%%\usepackage{xypic}

% Enumeration lists with different styles
%\usepackage{enumerate}

% Set up the theorem environments
%\newtheorem{thm}{Theorem}
%\newtheorem*{thm*}{Theorem}

\providecommand{\defnterm}[1]{\emph{#1}}

% The standard number systems
\newcommand{\complex}{\mathbb{C}}
\newcommand{\real}{\mathbb{R}}
\newcommand{\rat}{\mathbb{Q}}
\newcommand{\nat}{\mathbb{N}}
\newcommand{\intset}{\mathbb{Z}}

% Absolute values and norms
% Normal, wide, and big versions of the delimeters
\providecommand{\abs}[1]{\lvert#1\rvert}
\providecommand{\absW}[1]{\left\lvert#1\right\rvert}
\providecommand{\absB}[1]{\Bigl\lvert#1\Bigr\rvert}
\providecommand{\norm}[1]{\lVert#1\rVert}
\providecommand{\normW}[1]{\left\lVert#1\right\rVert}
\providecommand{\normB}[1]{\Bigl\lVert#1\Bigr\rVert}

% Differentiation operators
\providecommand{\od}[2]{\frac{d #1}{d #2}}
\providecommand{\pd}[2]{\frac{\partial #1}{\partial #2}}
\providecommand{\pdd}[2]{\frac{\partial^2 #1}{\partial #2}}
\providecommand{\ipd}[2]{\partial #1 / \partial #2}

% Differentials on integrals
\newcommand{\dx}{\, dx}
\newcommand{\dt}{\, dt}
\newcommand{\dmu}{\, d\mu}

% Inner products
\providecommand{\ip}[2]{\langle {#1}, {#2} \rangle}

% Calligraphic letters
\newcommand{\sF}{\mathcal{F}}
\newcommand{\sD}{\mathcal{D}}

% Standard spaces
\newcommand{\Hilb}{\mathcal{H}}
\newcommand{\Le}{\mathbf{L}}

% Operators and functions occassionally used in my articles
\DeclareMathOperator{\D}{D}
\DeclareMathOperator{\linspan}{span}
\DeclareMathOperator{\rank}{rank}
\DeclareMathOperator{\lindim}{dim}
\DeclareMathOperator{\sinc}{sinc}
\DeclareMathOperator{\tr}{tr}

% Probability stuff
\newcommand{\PP}{\mathbb{P}}
\newcommand{\E}{\mathbb{E}}

\begin{document}
\subsection{Case of single space dimension}

Let $X_t$ be an It\^o process satisfying the stochastic
differential equation
\[
dX_t = \mu_t \, dt + \sigma_t \, dW_t\,,
\]
with $\mu_t$ and $\sigma_t$ being adapted processes,
adapted to the same filtration as the Brownian motion $W_t$. 
Let $f$ be a function with continuous partial derivatives
$\pd{f}{t}$, $\pd{f}{x}$ and $\pdd{f}{x^2}$.

Then $Y_t = f(X_t)$ is also an It\^o process, and its stochastic
differential equation
is
\begin{align*}
dY_t &= \pd{f}{t} \, dt + \pd{f}{x} \, dX_t + \frac12 \pdd{f}{x^2} (dX_t)(dX_t)
\\
&= \left( \pd{f}{t} + \pd{f}{x} \mu_t + \frac12 \sigma_t^2 \right) \, dt
+ \pd{f}{x} \sigma_t \, dW_t\,,
\end{align*}
where all partial derivatives are to be taken at $(t,X_t)$.

\subsection{Case of multiple space dimensions}

There is also an analogue for multiple space dimensions.

Let $X_t$ be a $\real^n$-valued It\^o process satisfying the stochastic
differential equation
\[
dX_t = \mu_t \, dt + \sigma_t \, dW_t\,,
\]
with $\mu_t$ and $\sigma_t$ being adapted processes,
adapted to the same filtration as the $m$-dimensional Brownian motion $W_t$. 
$\mu_t$ is $\real^n$-valued and $\sigma_t$ is $L(\real^m, \real^n)$-valued.

Let $f\colon \real^n \times \real \to \real$ be a function with 
continuous partial derivatives.

Then $Y_t = f(X_t)$ is also an It\^o process, and its stochastic
differential equation
is
\begin{align*}
d Y_t 
      &= \pd{f}{t} \, dt + (\D f ) \, dX_t 
                + \tfrac12 dX_t^* ( \D^2 f ) dX_t \\
      &= \pd{f}{t} \, dt + (\D f ) \mu_t \, dt 
                + (\D f) \sigma_t \, dW_t 
                + \tfrac12 dW_t^* \, \sigma_t^* ( \D^2 f ) \sigma_t \, dW_t \\
      &= \pd{f}{t} \, dt + (\D f) \mu_t \, dt 
                + (\D f) \sigma_t \, dW_t 
                + \tfrac12 \tr\bigl( \sigma_t^* \,
                                  (\D^2 f) \, \sigma_t \bigr) \, dt \\
      &= \left( \pd{f}{t} + (\D f) \mu_t
                + \tfrac12 \tr\bigl( (\sigma_t \sigma_t^* ) (\D^2 f) \bigr) 
         \right) \, dt 
                + (\D f) \sigma_t \, dW_t \,,
\end{align*}
where 
\begin{itemize}
\item
$\tr$ is the trace operation; ${}^\ast$ is the transpose
\item
$\D f$ is the derivative with respect to the space variables;
its value is a linear transformation from $L(\real^n, \real)$
\item
$\D^2 f$ is the second derivative with respect to space variables;
represented as the Hessian matrix
\item
the third line follows because $dW_t^i \, dW_t^j = \delta_{ij} \, dt$.
\end{itemize}

The quadratic form $\tr(\sigma_t \sigma_t^* \, \D^2 f) \, dt $
represents the quadratic variation of the process.  When $\sigma_t$ is the identity
transformation, this reduces to the Laplacian of $f$.

It\^o's formula in multiple dimensions can also be written with
the standard vector calculus operators.
It is in the similar notation typically used for the 
related parabolic partial differential equation
describing an It\^o diffusion:

\[
d Y_t 
      = \left( \pd{f}{t} + \mu_t \cdot \nabla f
                + \tfrac12 \bigl( \nabla \cdot (\sigma_t \sigma_t^*) \nabla \bigr) f
         \right)  dt 
                + ( \sigma_t \, dW_t ) \cdot \nabla f \,.
\]


\begin{thebibliography}{9}
\bibitem{Oksendal}
Bernt \O{}ksendal.
{\em \PMlinkescapetext{Stochastic Differential Equations},
An Introduction with Applications}. 5th ed., Springer 1998.
\bibitem{Kuo}
Hui-Hsiung Kuo. {\em Introduction to Stochastic Integration}.
Springer 2006.
\end{thebibliography}

%%%%%
%%%%%
\end{document}
