\documentclass[12pt]{article}
\usepackage{pmmeta}
\pmcanonicalname{QuadraticVariationOfBrownianMotion}
\pmcreated{2013-03-22 18:41:25}
\pmmodified{2013-03-22 18:41:25}
\pmowner{gel}{22282}
\pmmodifier{gel}{22282}
\pmtitle{quadratic variation of Brownian motion}
\pmrecord{6}{41442}
\pmprivacy{1}
\pmauthor{gel}{22282}
\pmtype{Theorem}
\pmcomment{trigger rebuild}
\pmclassification{msc}{60H10}
\pmclassification{msc}{60J65}
%\pmkeywords{Brownian motion}
%\pmkeywords{quadratic variation}
\pmrelated{QuadraticVariation}

\endmetadata

% almost certainly you want these
\usepackage{amssymb}
\usepackage{amsmath}
\usepackage{amsfonts}

% used for TeXing text within eps files
%\usepackage{psfrag}
% need this for including graphics (\includegraphics)
%\usepackage{graphicx}
% for neatly defining theorems and propositions
\usepackage{amsthm}
% making logically defined graphics
%%%\usepackage{xypic}

% there are many more packages, add them here as you need them

% define commands here
\newtheorem*{theorem*}{Theorem}
\newtheorem*{lemma*}{Lemma}
\newtheorem*{corollary*}{Corollary}
\newtheorem*{definition*}{Definition}
\newtheorem{theorem}{Theorem}
\newtheorem{lemma}{Lemma}
\newtheorem{corollary}{Corollary}
\newtheorem{definition}{Definition}

\begin{document}
\PMlinkescapeword{property}
\PMlinkescapeword{interval}

\begin{theorem*}
Let $(W_t)_{t\in\mathbb{R}_+}$ be a standard Brownian motion. Then, its quadratic variation exists and is given by
\begin{equation*}
[W]_t=t.
\end{equation*}
\end{theorem*}

As Brownian motion is a martingale and, in particular, is a semimartingale then \PMlinkname{its quadratic variation must exist}{QuadraticVariationOfASemimartingale}. We just need to compute its value along a sequence of partitions.

If $P=\{0=t_0\le t_1\le\cdots\le t_m=t\}$ is a \PMlinkname{partition}{SubintervalPartition} of the interval $[0,t]$, then the quadratic variation on $P$ is
\begin{equation*}
[W]^P=\sum_{k=1}^m(W_{t_k}-W_{t_{k-1}})^2.
\end{equation*}
Using the property that the increments $W_{t_k}-W_{t_{k-1}}$ are independent normal random variables with mean zero and variance $t_k-t_{k-1}$, the mean and variance of $[W]^P$ are
\begin{align*}
\mathbb{E}\left[[W]^P\right]&=\sum_{k=1}^m\mathbb{E}\left[(W_{t_k}-W_{t_{k-1}})^2\right]=\sum_{k=1}^m(t_k-t_{k-1})=t,\\
\operatorname{Var}\left[[W]^P\right]&=\sum_{k=1}^m\operatorname{Var}\left[(W_{t_k}-W_{t_{k-1}})^2\right]=\sum_{k=1}^m2(t_k-t_{k-1})^2\\
&\le 2|P|\sum_{k=1}^m(t_k-t_{k-1})=2|P|t.
\end{align*}
Here, $|P|=\max_k(t_k-t_{k-1})$ is the mesh of the partition.
If $(P_n)_{n=1,2,\ldots}$ is a sequence of partitions of $[0,t]$ with mesh going to zero as $n\rightarrow \infty$ then,
\begin{equation*}
\mathbb{E}\left[([W]^{P_n}-t)^2\right]\le 2|P_n|t\rightarrow 0
\end{equation*}
as $n\rightarrow\infty$. This shows that $[W]^{P_n}\rightarrow t$ in the $L^2$ norm and, in particular, converges in probability. So, $[W]_t=t$.

%%%%%
%%%%%
\end{document}
