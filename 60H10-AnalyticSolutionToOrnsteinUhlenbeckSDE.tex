\documentclass[12pt]{article}
\usepackage{pmmeta}
\pmcanonicalname{AnalyticSolutionToOrnsteinUhlenbeckSDE}
\pmcreated{2013-03-22 17:19:29}
\pmmodified{2013-03-22 17:19:29}
\pmowner{stevecheng}{10074}
\pmmodifier{stevecheng}{10074}
\pmtitle{analytic solution to Ornstein-Uhlenbeck SDE}
\pmrecord{4}{39676}
\pmprivacy{1}
\pmauthor{stevecheng}{10074}
\pmtype{Derivation}
\pmcomment{trigger rebuild}
\pmclassification{msc}{60H10}
\pmclassification{msc}{60-00}

% The standard font packages
\usepackage{amssymb}
\usepackage{amsmath}
\usepackage{amsfonts}

% For neatly defining theorems and definitions
%\usepackage{amsthm}

% Including EPS/PDF graphics (\includegraphics)
%\usepackage{graphicx}

% Making matrix-based graphics
%%%\usepackage{xypic}

% Enumeration lists with different styles
%\usepackage{enumerate}

% Set up the theorem environments
%\newtheorem{thm}{Theorem}
%\newtheorem*{thm*}{Theorem}

\providecommand{\defnterm}[1]{\emph{#1}}

% The standard number systems
\newcommand{\complex}{\mathbb{C}}
\newcommand{\real}{\mathbb{R}}
\newcommand{\rat}{\mathbb{Q}}
\newcommand{\nat}{\mathbb{N}}
\newcommand{\intset}{\mathbb{Z}}

% Absolute values and norms
% Normal, wide, and big versions of the delimeters
\providecommand{\abs}[1]{\lvert#1\rvert}
\providecommand{\absW}[1]{\left\lvert#1\right\rvert}
\providecommand{\absB}[1]{\Bigl\lvert#1\Bigr\rvert}
\providecommand{\norm}[1]{\lVert#1\rVert}
\providecommand{\normW}[1]{\left\lVert#1\right\rVert}
\providecommand{\normB}[1]{\Bigl\lVert#1\Bigr\rVert}

% Differentiation operators
\providecommand{\od}[2]{\frac{d #1}{d #2}}
\providecommand{\pd}[2]{\frac{\partial #1}{\partial #2}}
\providecommand{\pdd}[2]{\frac{\partial^2 #1}{\partial #2}}
\providecommand{\ipd}[2]{\partial #1 / \partial #2}

% Differentials on integrals
\newcommand{\dx}{\, dx}
\newcommand{\dt}{\, dt}
\newcommand{\dmu}{\, d\mu}

% Inner products
\providecommand{\ip}[2]{\langle {#1}, {#2} \rangle}

% Calligraphic letters
\newcommand{\sF}{\mathcal{F}}
\newcommand{\sD}{\mathcal{D}}

% Standard spaces
\newcommand{\Hilb}{\mathcal{H}}
\newcommand{\Le}{\mathbf{L}}

% Operators and functions occassionally used in my articles
\DeclareMathOperator{\D}{D}
\DeclareMathOperator{\linspan}{span}
\DeclareMathOperator{\rank}{rank}
\DeclareMathOperator{\lindim}{dim}
\DeclareMathOperator{\sinc}{sinc}

% Probability stuff
\newcommand{\PP}{\mathbb{P}}
\newcommand{\E}{\mathbb{E}}

\begin{document}
This entry derives
the analytical solution
to the stochastic differential equation
for the Ornstein-Uhlenbeck process:
\begin{align}\label{eq:x}
dX_t = \kappa ( \theta - X_t) \, dt + \sigma \, dW_t\,,
\end{align}
where $W_t$ is a standard Brownian motion,
and $\kappa > 0$, $\theta$, and $\sigma > 0$ are
constants.

Motivated by the observation
that $\theta$ is supposed to be the long-term mean
of the process $X_t$,
we can simplify the SDE \eqref{eq:x}
by introducing the change of variable 
\[
Y_t = X_t - \theta
\]
that subtracts off the mean.
Then $Y_t$ satisfies the SDE:
\begin{align}\label{eq:y}
dY_t = dX_t = -\kappa Y_t \, dt + \sigma \, dW_t\,.
\end{align}

In SDE \eqref{eq:y}, the process $Y_t$ is seen to have a drift
towards the value zero, at an exponential rate $\kappa$.  This motivates
the change of variables
\[
Y_t = e^{-\kappa t} Z_t \quad \Leftrightarrow \quad Z_t = e^{\kappa t} Y_t\,,
\]
which should remove the drift.
A calculation with the product rule for It\^o integrals
shows that this is so:
\begin{align*}
dZ_t &= \kappa e^{\kappa t} Y_t \, dt + e^{\kappa t} \, dY_t \\
&= \kappa e^{\kappa t} Y_t \, dt + 
e^{\kappa t} \bigl( -\kappa Y_t \, dt + \sigma \, dW_t \bigr) \\
&=
0 \, dt + \sigma e^{\kappa t} \, dW_t \,.
\end{align*}
The solution for $Z_t$ is immediately obtained 
by It\^o-integrating both sides from $s$ to $t$:
\begin{align*}
Z_t = Z_s + \sigma \int_s^t e^{\kappa u} \, dW_u\,.
\end{align*}
Reversing the changes of variables, we have:
\[
Y_t = e^{-\kappa t} Z_t =
e^{-\kappa (t-s)} Y_s
+ \sigma e^{-\kappa t} \int_s^t e^{\kappa u} \, dW_u\,,
\]
and
\[
X_t = Y_t + \theta = \theta + e^{-\kappa (t-s)} (X_s - \theta)
 + \sigma \int_s^t e^{-\kappa(t-u)} \, dW_u\,.
\]


%%%%%
%%%%%
\end{document}
