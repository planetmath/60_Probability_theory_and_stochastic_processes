\documentclass[12pt]{article}
\usepackage{pmmeta}
\pmcanonicalname{GaussianProcess}
\pmcreated{2013-03-22 15:22:48}
\pmmodified{2013-03-22 15:22:48}
\pmowner{CWoo}{3771}
\pmmodifier{CWoo}{3771}
\pmtitle{Gaussian process}
\pmrecord{5}{37209}
\pmprivacy{1}
\pmauthor{CWoo}{3771}
\pmtype{Definition}
\pmcomment{trigger rebuild}
\pmclassification{msc}{60G15}
\pmclassification{msc}{60G60}
\pmsynonym{Gaussian random field}{GaussianProcess}
\pmdefines{normal process}

\endmetadata

\usepackage{amssymb,amscd}
\usepackage{amsmath}
\usepackage{amsfonts}

% used for TeXing text within eps files
%\usepackage{psfrag}
% need this for including graphics (\includegraphics)
%\usepackage{graphicx}
% for neatly defining theorems and propositions
%\usepackage{amsthm}
% making logically defined graphics
%%%\usepackage{xypic}

% define commands here
\begin{document}
\PMlinkescapeword{gaussian}

A stochastic process $\lbrace X(t)\mid t\in T\rbrace$ is said to be
a \emph{Gaussian process} if all of the members of its f.f.d.
(family of finite dimensional distributions) are joint normal
distributions.  In other words, for any positive integer $n<\infty$,
and any $t_1,\ldots,t_n\in T$, the joint distribution of random
variables $X(t_1),\ldots,X(t_n)$ is jointly normal.

As an example, any Wiener process is Gaussian.

\textbf{Remark}.  Sometimes, a Gaussian process is known as a \emph{Gaussian random field} if $T$ is a subset, usually an embedded manifold, of $\mathbb{R}^m$, with $m>1$.
%%%%%
%%%%%
\end{document}
